\documentclass[a4paper,12pt]{article}
\usepackage[utf8]{inputenc}

%opening
\title{Empirically finding the value of absolute zero using the ideal gas law}
\author{Terry Qi}

\begin{document}

\maketitle

\section{Design}
\subsection{Introduction}
During a school trip for the purpose of completing my duke of edinburgh, my supervisor had told me a little fact about  


\subsection{Research Question}
\begin{quote}
    What is the relationship between the pressure of a (specific) gas against its temperature under constant volume and quantity?
\end{quote}

\subsection{Background}



\subsection{Variables}
\paragraph{Independent Variable}
The temperature of the gas in unit Kelvin

\paragraph{Depended Variable}
The pressure of the gas in unit Pascals

\subsection{Control Variables}

% table here

\subsection{Materials}

\begin{itemize}
    \item Vernier gas pressure sensor \& Connection hub
    \item laptop with Data logger lite software
    \item 500ml glass beaker
    \item digital thermometer
    \item Bunsen burner and tripod
\end{itemize}

\subsection{Method}


\begin{enumerate}
    \item Connect the sensor \& connection hub \& laptop with software, change pressure readings
    \item Setup the experiment as shown in the

\end{enumerate}


\subsection{Diagrams}
\subsection{Safety}

\section{Data}
\subsection{Raw Data}
\subsection{Processing}

\section{Conclusions}
\subsection{Result}
\subsection{Implications}
\subsection{Reflection}

\end{document}
