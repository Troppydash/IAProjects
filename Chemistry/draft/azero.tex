\documentclass[a4paper,12pt]{article}
\usepackage[a4paper,margin=1in,footskip=0.25in]{geometry}
\usepackage[utf8]{inputenc}

% science
\usepackage{amsmath}
\usepackage{array}
\usepackage{siunitx}

% layout
\usepackage{float}
\usepackage{parskip}
\usepackage{graphicx}
\usepackage{circuitikz}
\usepackage{longtable}
\usepackage{hyperref}
\usepackage{subfig}


\title{Empirically finding the value of absolute zero using the ideal gas law}
\author{Terry Qi}

\begin{document}

\maketitle

\section{Design}
\subsection{Introduction}
During a school trip for the purpose of completing my duke of edinburgh, my supervisor had told me a little fact about    


\subsection{Research Question}
\begin{quote}
    What is the relationship between the pressure of a (specific) gas against its temperature under constant volume and quantity?
\end{quote}

\subsection{Background}



\subsection{Variables}
\paragraph{Independent Variable}
The temperature of the air mixture in unit Celsius

\paragraph{Depended Variable}
The pressure of the air mixture in unit Pascal

\subsection{Control Variables}

% table here

\subsection{Materials}

\begin{itemize}
    \item Gas pressure measurement probe ($\pm 0.25\si{psi}$)
    \item 500ml glass beaker
    \item 2 $\times$ digital thermometer ($\pm 0.1\si{C}$)
    \item 2 $\times$ retort stands
    \item bunsen burner, tripod, and heating mat
    \item lighter
\end{itemize}

\subsection{Method}


\begin{enumerate}
    \item Setup the experiment as shown as in figure ()
    \item 

\end{enumerate}


\subsection{Diagrams}
\subsection{Safety}

\section{Data}
\subsection{Raw Data}
\subsection{Processing}

\section{Conclusions}
\subsection{Result}
\subsection{Implications}
\subsection{Reflection}

\end{document}
