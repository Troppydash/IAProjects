\documentclass[a4paper,12pt]{article}
\usepackage[a4paper,bindingoffset=0.2in,%
left=0.75in,right=0.75in,top=0.75in,bottom=0.75in,%
footskip=.25in]{geometry}
\usepackage[utf8]{inputenc}
\usepackage{lineno}
\usepackage{refcount}
\usepackage{parskip}
\usepackage{textcomp}

% this generate references
\newcommand*{\test}[2]{%
    \ifnum\getrefnumber{#1}=\getrefnumber{#2}%
    \ref{#1}%
    \else
    \ref{#1}--\ref{#2}%
    \fi
}
\renewcommand\linenumberfont{\normalfont\bfseries\small}


\newcommand{\arr}{$\,\to\,$}
\begin{document}

\paragraph{Topic}
How creative and persuasive text explore the issue of slavery - Culture, Identity, and Community

\paragraph{Literary Text}
Perfume, The Story of a Murder, by Patrick Su\"skind

\begin{linenumbers}
    From his first glance at Monsieur Grimal --- no, from the first breath that sniffed in the odor
enveloping Grimal --- Grenouille knew that this man was capable of thrashing him to death for the least
infraction. His life was worth precisely as much as the work he could accomplish and consisted only of
whatever utility Grimal ascribed to it.

And so, Grenouille came to heel, never once making an attempt
to  resist.  With  each  new  day,  he  would  bottle  up  inside  himself  the  energies  of  his  defiance  and
contumacy  and  expend  them  solely  to  survive  the  impending  ice  age  in  his  ticklike  way.  Tough,
uncomplaining,  inconspicuous,  he  tended  the  light  of  life’s  hopes  as  a  very  small,  but  carefully
nourished flame. He was a paragon of docility, frugality, and diligence in his work, obeyed implicitly,
and appeared satisfied with every meal offered. In the evening, he meekly let himself be locked up in a
closet off to one side of the tannery floor, where tools were kept and the raw, salted hides were hung.
There  he  slept  on  the  hard,  bare  earthen  floor.  During  the  day  he  worked  as  long  as  there  was
light-eight  hours  in  winter,  fourteen,  fifteen,  sixteen  hours  in  summer.  He  scraped  the  meat  from
bestially  stinking  hides,  watered  them  down,  dehaired  them,  limed,  bated, and fulled  them,  rubbed
them down with pickling dung, chopped wood, stripped bark from birch and yew, climbed down into
the tanning pits filled with caustic fumes, layered the hides and pelts just as the journeymen ordered
him,  spread  them  with  smashed  gallnuts,  covered  this  ghastly  funeral  pyre  with  yew  branches  and
earth.  Years  later,  he  would  have  to  dig  them  up  again  and  retrieve  these  mummified  hide
carcasses-now tanned leather --- from their grave.

When he was not burying or digging up hides, he was hauling water. For months on end, he
hauled  water  up  from  the  river,  always  in  two  buckets,  hundreds  of  bucketfuls  a  day,  for  tanning
requires vast quantities of water, for soaking, for boiling, for dyeing. For months on end, the water
hauling left him without a dry stitch on his body; by evening his clothes were dripping wet and his skin
was cold and swollen like a soaked shammy.
\end{linenumbers}

\newpage

\paragraph{Language Text} What to the slave is the Fourth of July? By Frederick Douglass

https://www.theguardian.com/commentisfree/2018/jul/05/frederick-douglass-anti-slavery-speech-what-to-the-slave-is-the-fourth-of-july

\resetlinenumber
\begin{linenumbers}
 What, to the American slave, is your Fourth of July? I answer: a day that reveals to him, more than all other days in the year, the gross injustice and cruelty to which he is the constant victim. To him, your celebration is a sham; your boasted liberty, an unholy license; your national greatness, swelling vanity; your sounds of rejoicing are empty and heartless; your denunciations of tyrants, brass-fronted impudence; your shouts of liberty and equality, hollow mockery; your prayers and hymns, your sermons and thanksgivings, with all your religious parade, and solemnity, are, to him, mere bombast, fraud, deception, impiety, and hypocrisy – a thin veil to cover up crimes which would disgrace a nation of savages. There is not a nation on the earth guilty of practices, more shocking and bloody, than are the people of these United States, at this very hour.

 Go where you may, search where you will, roam through all the monarchies and despotisms of the old world, travel through South America, search out every abuse, and when you have found the last, lay your facts by the side of the everyday practices of this nation, and you will say with me, that, for revolting barbarity and shameless hypocrisy, America reigns without a rival.

 Americans! Your republican politics are flagrantly inconsistent. The existence of slavery in this country brands your republicanism as a sham, your humanity as a base pretense, and your Christianity as a lie. It destroys your moral power abroad; it corrupts your politicians at home. It saps the foundation of religion; it makes your name a hissing, and a byword to a mocking earth. It is the antagonistic force in your government, the only thing that seriously disturbs and endangers your Union. It fetters your progress; it is the enemy of improvement, the deadly foe of education; it fosters pride; it breeds insolence; it promotes vice; it shelters crime; it is a curse to the earth that supports it; and yet, you cling to it, as if it were the sheet anchor of all your hopes. Oh! Be warned! Be warned! A horrible reptile is coiled up in your nation’s bosom; the venomous creature is nursing at the tender breast of your youthful republic; for the love of God, tear away, and fling from you the hideous monster, and let the weight of twenty millions crush and destroy it forever!

 Allow me to say, in conclusion, notwithstanding the dark picture I have this day presented of the state of the nation, I do not despair of this country. There are forces in operation, which must inevitably work the downfall of slavery. “The arm of the Lord is not shortened,” and the doom of slavery is certain. I, therefore, leave off where I began, with hope.
\end{linenumbers}

\newpage

\paragraph{Notes}

\paragraph{Language Text}

Author: Escaped slavery, became national leader of movement in Massachusetts and New Work, famous for speaking. As a living counterexample to the arguments at the time in the inability of slaves to conceive intellectual capacity. Northerners surprised he was a slave.

During the civil war, Douglass argued for the men of color to enlist, quoting a witness of black confederates with muskets.

Background: Recorded on the 5th of July in 1852, indoors in the state of New York. The state is a center of slavery in the 1700, with more than 2/5 households owning slaves. The speech is conducted after the ending of the civil war, and Lincoln's death with the 13th amendment, where racial tension is still high.

\paragraph{Language Techniques}
Writer's purpose to advocate for change, a persuasive speech. Mainly targeting towards law makers \& political personnel at the time, as well as common citizens. Focuses on the idea of change.

Broadly, directly addressing the listener, interactively challenging the opposition, personification of the topic/nation and allegory to the bible --- towards religion, repetition of the personnel pronouns, irony in the strength of the nation. Rhetorical start \& hopeful end.

Rhetorical question [1]: aim of section, reflection of the listener,  increasing engagement with two sensitive topics, thus reaching further with this speech.

Personal Pronoun [3-5]: talking to you particularly, creating personal connections and that ``you'' are guilty for this crime, more compelling to change under direct pressure.

Irony [16]: increase guilt in context, while appealing out of context. this mocks the very nation of the listeners, causing shame upon them and a pressure to change, additionally, making slavery unjustifiable and following the line of thoughts of ``America first''.

Allegory [26-28]: the venomous creature --- a snake, to the biblical the garden of Eden, creating familiarity with the christian population of America, the hope is that knowledged people will understand the consequence of the analogy, therefore to take action.

\paragraph{Literary Text} Author: A German writer, likes being alone and away from media attention. living after WWII in the 20th century.

Background: Romantic period, advances in science at the cost of nature. Industrialization is a necessary step to advancement, but at a cost of others and the environment --- the popularization of slavery. Social norms to it are born, and had only recently changed --- Atlantic slave trade. Massive inequality in Europe, but also developing.

\paragraph{Language Techniques}
Writer's purpose to display the uneasiness of industrialization, displaying the treatment of people as the social norm to be of disgust. The tone and register of the narrator makes this encounter very natural and common. Mainly targeting towards fictional fantasy readers, who enjoys a mix of fiction and reality.

Broadly, multiple times of comparing Grenouille as an animal, to a worker, and to the product he is making. Powerful adjectives usages, out of context sometimes positive connotation, sometimes negative. A sense of irony in ``digging its own grave''. Reader have an unease feeling, both in the casualness of the narrator, but also in the the event it describes so vividly using adjectives.

Adjectives [9] [12][14-16]: more realistic objects, showcasing a negative connotation shift over time of his labor. sympathetic the character with the audience, initial personality setting as with the norm of people in the era, displaying the reality of the genre.

Irony [19-21]: that Grenouille is preparing and digging up the dead leather, just like what the forced labor is doing. Showcasing the crude reality in contrasting to the previous fantasy section, communicating with the reader of the sudden shift in events, encouraging moral reflection along with the fictional Grenouille.

Metaphor [7]: letting Grenouille be a tick in the ice-age, usage of metaphors of Grenouille with creatures --- tick, frog. Exaggerating Grenouille's situation, sympathizing with the villain. Analogy to that of an animal to signify weakness and helpless, early in the growth stage. Connection and sympathy towards the helpless animal, albeit with a negative connatation --- leaching ticks, edible frogs.

\newpage
\section{Format}

Topic: Culture, Identity, and Community, ``How creative and persuasive text describes slavery?''

``Perfume, the story of a murderer'', from Patrick S\"uskind, 1985.

``What to the slave is the Fourth of July?'', by Frederick Douglass, 05/07/1852


% TODO: change lines
\begin{itemize}
 \item Text1, Text2, GI, reasons, KA: how social \& economic issues contributes to the problem of slavery, answer question
 % perfume
 \item Realistic fantasy, era. Description. Widely, unappetizing descriptions \arr author uneasiness towards indust. era.
 % adjectives
 Strong specific adjective. positive adj [9], socially accepted form. but connotations depreciates over extract, from [12] to [14], progression hides as analogy of G over slavery, author perception of cost of indust. , KA: social cost of slavery. Ties in with realism, contrast with previous fiction, striking image.
 \item line [19] ironic, cliche to ``digging own grave'', crude reality interpreted author, reinforced metaphor [24] comparing G's body to hide, lack of difference between slave \& resource. KA: economical issue, FOP, [3-4]. Wide reflection peek Atlantic trade, money, child labor. Pause \& reflect moral stance.
 \item Metaphor [7] G as tick slavery as ice-age. Exaggerate power, ties to adjectives exaggerate environment, powerless against natural disaster, like G. Sympathizing, not factual: meaning \& intention, contrast science approach of perfumes. Develop G as paragon, make perfume creatively.
 \item Context in decrease appreciation of human creativity, industrial revol. as turning point. Differ in motivations \& audience, approach to KA \& GI depend on context.

 % fourth of july
 \item motivational speech, extract expressing distress \& reform hope within US society. rhetorical [1], aim, structure, signal to move on. Sensitive topic ``slave'' ``04/07``, create emotional response of audience \& attention, curiosity relationship. KA upper class able change.
 \item Personal pronoun, singles out. attack beliefs [3-4], establish guilt for crime, while benefiting, repetition pile up social pressure. Pathos in deconstruction values induce anger, analogous to destroy \& decompose social value slaves. KA: Human rights as value but not reinforce, double standard.
 \item Anger direct to personalization [18], slavery as being, contagious disease, common target. Contrast ``abroad'' ``at home'' for wide effects, not end at victims, spread  contaminates, globalization [23]. Personalization \arr urgency, responsible.
 \item Context: era, location epicenter, peak 2/5 households own

 % thoughts
 \item Perfume, slavery perceived cost industrial, mock worker [13]. contrast speech slavery unfortunate illness, ignorance host. Literary focus adj, adv, character to life, audience sympathize; speech personal pronouns, address to guilt, pressure actions. Creative: story telling meaning, persuasive: emotion sympathize victims.
\end{itemize}

\newpage
\section{Speech}
My literary extract is Perfume, ``the story of a murderer'' from author Patrick Suskind, written and published in the 20th century. My language text is an extract of the speech ``What to the slave is the Fourth of July?'' by Frederick Douglass, performed in the 19th century. I will be exploring the global issue of Culture, Identity, and Community, because I have previously explored language texts around social issues, and wanted to compare it to a more literary text, such as Perfume. The key area I'll be exploring is the social and economic issues of industrialization, and I hope to answer the question of ``How creative and persuasive text describe slavery?''.

% perfume
The first extract Perfume, is a realistic fantasy novel, set in the 18th century romantic era central France. In the extract Grenouille is seen to be sold to a tanner Monsieur Grimal after his time in the orphanage, and it describes his next years as a young worker. More widely speaking, the unappetizing descriptions of the social and environmental settings connect to Suskind's uneasiness towards industrialization.
% adjectives
Most visible are the uses of strong and specific adjectives throughout the extract. Positive adjectives on line 9 --- describing Grenouille as the ``paragon of docility, frugality, and diligence'' frames the socially accepted form of a worker. But the connotations of the adjective depreciates over the course of the extract, from the ``raw, salted hides'' on line 12, to a ``bestially stinking hides'' on line 15, using the hides as an analogy of Grenouille as he persists through slavery, and ultimately stating the author's perception of the cost of industrialization. This references my key area, for it explores one of the multiple interpretation of the issues with the industry era. This also ties-in with the realistic part of the novel, and contrast with the fictional previous section of the nursery, creating a striking image in the readers to be memorized.
% irony
The section from line 19 to 21, stating ``he would have to dig them up again and retrieve these mummified hides'', comes up to be ironic. The digging of the hides is a analogy of a slave digging his own grave, showcasing the crude reality as interpreted by the author, which is further reinforced later in line 25, ``the water hauling left him without a dry stitch on his body'', comparing Grenouille's body with the hides he is soaking, connecting to the idea that there is a lack of differentiation between slaves and resources. This ties into the economical issues of slavery, that labor is treated as a factor of production, for on line 3 to 4, Grenouille's life is ``worth precisely as much as the work he could accomplish''. Against the wider world, this is a reflection on the existence and peek of the Atlantic slave trade in the 18th century, where adults and children alike were employed for the sake of production. This use of irony allows the reader to pause and reflect on their moral stances upon this issue of slavery.
% metaphor
Metaphors are used in associating Grenouille with various animals. Such as on line 7, ``to expend them solely to survive the impending ice age in his ticklike way'', associating Grenouille as a tick and the slavery as an ice age. This ties back to the use of adjectives in exaggerating the power difference between Grenouille and his owner, for a tick is powerless against a natural disaster, like how the young character has no power in the extract. Additionally, the naming of Grenouille --- french word for frog, connects to this aspect of power. Each chapter corresponds to the growth stages of the character: from an egg, to a tadpole, than to a maturing tadpole, and finally as an adult frog. This is seen as a rejection of the systematic view of life, but seeking to represent growth through natural processes --- a key idea in romanticism. This links back to the analogy on line 12 and 15, comparing the process of slavery to that of a hide being buried, easing the interpretation of slavery as a natural process. This ultimately leads to the reader sympathizing with the character, showcasing that factual information in the form of statistics is not necessary required in areas of literature and creative works --- something the wider text disagrees with for its scientific approach to perfume making.
% idea
This is important in developing Grenouille as the paragon of creativity and artistic. He later approached perfume making not through measurement and theory, but through the imagination of the mind. The context of this is significant in that it showcases the decrease in the appreciation to human creativity over time, marking the industrial revolution as the turning point of creative works and human imagination. The avocation of modern slavery is commonly associated with numbers and listings, ironically becoming more inline with the concept of treating people as mere numbers rather than a living animals.
% link
The two extracts differ in their motivations and target audience, therefore the approach to the global issue is dependence upon the context where it is raised and evaluated against.

% speech
The speech ``What to the slave is the Fourth of July?'', performed in the 19th century by antislavery activist Frederick Douglass is an extract of a persuasive speech expressing Douglass's distress over the American attitude towards slavery, and a hope for reform.
% rehtorical question
The speech starts with a rhetorical question on line 1: ``What to the slave is the Fourth of July?'', clearly states the aim of this extract, establishing a structure and giving the signal for the listeners to move-on from the previous point. Additionally, the choice of the expression ``slave'' and ``Fourth of July'' marks two sensitive topics, creating a emotion response that gains the attention of the audience full of law-makers, politicians, and decision makers, forming curiosity. This connects to the key concept in acknowledging the upper-class who are able to conduct economical and social change.
% personal pronoun
This theme of direct addressing is apparent throughout the speech. Personal pronouns are used to answer the rhetrical question posed on line 1, quote ``To him, your celebration is a sham'' on line 3. This establishes idea that the listener is guilty for the crime of slavery, and continuing to pile up this pressure through repetition in attacking the beliefs of the American citizens, on line 4: ``your boasted liberty, an unholy license; your national greatness, swelling vanity''. This usage of pathos in deconstructing the values of the listeners serves to induce anger, but is very appropriate. It references the social destruction of slavery on its victims, and flipping the edges of the dagger onto the upper-middle class audience. Human rights is a value at the time, but it is us that creates the double standard and fails to take actions to enforce it, and for what?
% personalization
This idea of the personal involvement of the person on slavery leads to the personalization of it on line 19, ''It destroys your moral power abroad, it corrupts your politicians at home''. This solidates the existence of slavery as a terrible being, formulating it as a spreading diseases that should be a common targets of social groups alike. The significance of the contrast of ``abroad'' and ``at home'' is to show the wide effects of the plague, that the issue of slavery does not end at its victims, it spreads, it contaminates, a globalization that consider on line 24-25, ``it is a curse to the earth that supports it''. Furthermore, the analogy of slavery to a reptile marks the disease to be a leach. Slavery needs to end at this moment, and we are responsible for all of its consequences.
% compare
The context of the speech is also crucial. For this is a period of the peak in slavery, with NYC being the epicenter of this disease --- more than 2/5 households owns slave in the era.
% wider works?


% final thoughts
Some final thoughts, in Perfume, slavery is used depict the author's perceived cost of the industrial age, it mocks the modern workers on line 14: ``eight hours in the winter, fourteen, fifteen, sixteen hours in the summer''. This contrasts to the speech for it viewed slavery as an unfortunate illness, for it is the ignorance of the host that it persists. While the literary text have a focus on the adjectives and adverbs, making the fictional character come to life; the speech excels in its use of personal pronouns, directly addressing the audience into guilt, causing actions to be taken against the subject. Showing that creative texts use the element of story telling to bring meaning, while persuasive texts uses emotions to bring the audience to sympathize towards the victims.



\end{document}
