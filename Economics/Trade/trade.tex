\documentclass[a4paper,12pt]{article}
\usepackage[utf8]{inputenc}
\usepackage{graphicx}
\usepackage{amsmath}
\usepackage{float}
\usepackage{parskip}
\usepackage{varwidth}

% TIKZ
\usepackage{physics}
\usepackage{amsmath}
\usepackage{tikz}
\usepackage{mathdots}
\usepackage{yhmath}
\usepackage{cancel}
\usepackage{color}
\usepackage{siunitx}
\usepackage{array}
\usepackage{multirow}
\usepackage{amssymb}
\usepackage{gensymb}
\usepackage{tabularx}
\usepackage{extarrows}
\usepackage{booktabs}
\usetikzlibrary{fadings}
\usetikzlibrary{patterns}
\usetikzlibrary{shadows.blur}
\usetikzlibrary{shapes}

% TIKZ Figure
%\newcommand{\sized}[2][0.7\textwidth]{\begin{varwidth}{#1}#2\end{varwidth}}
\newcommand{\tikzfig}[1]{\input{#1}}

\linespread{1.2}

% TEXCOUNT
\usepackage{moreverb} % for verbatim ouput
\newcommand{\initTexcount}[1]{\immediate\write18{texcount -inc -incbib
		-sum #1 > .wordcount.tex}}
\newcommand\wordcount{
		\verbatiminput{.wordcount.tex}}

\title{The consequences of interdependence through UK's import tariffs on Chinese exported Aluminum Extrusion}
\author{\vspace{-8ex}}
\date{\vspace{-8ex}}

\begin{document}

\maketitle
Word count: 822\\
Article: https://www.reuters.com/markets/commodities/uk-may-impose-duties-up-29-chinese-aluminium-extrusions-2022-05-20/, ``UK may impose duties of up to 29\% on Chinese aluminium extrusions'', published by Reuters, written on 21/05/2022, accessed 07/06/2022\\
Topic: The Global Economy\\
Key Concept: Interdependence

%TC:ignore
\initTexcount{trade.tex}
\wordcount
%TC:endignore

\newpage

\section*{Article}
``UK may impose duties of up to 29\% on Chinese aluminium extrusions''

LONDON, May 20 (Reuters) - Britain may impose anti-dumping duties of up to 29\% on aluminium extrusions from China to protect domestic producers, a trade agency said on Friday.

Aluminium extrusions - widely used in the transport, construction and electronics industries - are being dumped in Britain at lower prices than they are sold in China, the Trade Remedies Authority said in a interim report.

"The TRA determined that there is already damage to the UK industry, having found clear evidence of price undercutting, indicating that UK businesses are struggling to compete with the dumped imports," a statement said.

Provisional measures will be imposed as the TRA completes its investigation, requiring Chinese companies exporting to Britain to provide a bank guarantee beginning on May 28, it added.

Duties ranging from 7.3\% to 29.1\% were recommended, depending on the company and the level of dumping margin, the interim report said.

Three companies were named - Press Metal International Group, Shandong Nanshan (600219.SS) and Haomei Group - plus there were categories for other co-operating and non-cooperating exporters.

Press Metal International is a Chinese unit of Malaysia's Press Metal (PMET.KL).

The TRA was established after Britain left the European Union to investigate unfair trade practices and measures.

The aluminium extrusion investigation is the first one resulting from a British industry claiming unfair trade practices, the TRA said.

One major firm that produces aluminium extrusions in Britain is Norway's Norsk Hydro (NHY.OL)

\newpage
\section*{Commentary}

% STRUCTURE

% DEFINITION
The article describes Britain's possible imposition of tariffs upon imported Chinese aluminum extrusion as anti-dumping measures. Import tariffs are taxes imposed per unit good or service imported from a particular country --- the proposed tariff is proportional of maximum 29\% --- a protectionist policy aiming to reduce international interdependence and assist domestic firms through reduced competition. The argument is suspected dumping: the exporting of goods and services under their costs of production often caused by governmental subsidies, that was viewed by the TRA (Trade Remedies Authority) as ``unfair trade practices''. I will be evaluating the policy's effectiveness and its potential consequences for the two nations through the key concept of interdependence.

% State the addition of tariff, brief connection to intended purpose.
This trade protectionism is likely to benefit British Aluminum Extrusions producers in the short term at a cost of reduced international trade. A trade supply-demand diagram (figure \ref{fig:tariff}) can be used to explain this.

\begin{figure}[H]
	\centering
	 \tikzfig{assets/tariffs.txt}
	\caption{}
	\label{fig:tariff}
\end{figure}

The imposition of a proportional tariff on Chinese aluminum extrusions will increase their COP at all supplied quantities, causing an upwards shift from $S(China)$ to $S'(China)$. Assuming that the Chinese supply is perfectly elastic for its large Aluminum manufacturing sector (citation), the shift will increase domestic extrusions price from $P_w$ to $P_2$ as consumption decreases from $Q_{d1}$ to $Q_{d2}$;
%If the new price $P_2$ is below the equilibrium price where $D=S(Domestic)$
imports will drop from $Q_{d1}-Q_{1}$ to $Q_{d2}-Q_{2}$ decreasing Chinese-firm revenue from $J+I+H$ to $I$; British producers will increase production from $Q_1$ to $Q_2$ increasing their revenue from $F+G$ to $B+C+F+G+H$. The British government will collect revenue $D$ due to the tariff. Overall, the inefficiency of British producers and the decreased consumption both reduce welfare of size $C$ and $E$, a dead-weight loss of decreasing social surplus.

% Key concept
The key concept of interdependence is demonstrated in the interconnectivity between the Chinese and British economy during trade, including the interdependence between foreign producers and domestic consumers. The protectionist policy's reduction in social welfare and efficiency through comparative advantages of the nations is more apparent in this interdependent global economy.


% TODO: NEED THREE POINTS
% CONCLUSION, LONG,SHORT-TERM, ASSUMPTIONS, STAKEHOLDERS, PRO CON, PRIORITiES
% Point 1, tariff's effect on domestic firms and foreign firms.
The tariff will directly benefit domestic aluminum producers and increase government revenue that creates supply-side growth in the long-run. For the tariff has increased domestic producer production and revenues, it is likely to lift ``struggling'' domestic firms back into operation by decreasing the competitiveness of the subsidized extrusions within the domestic market. More domestic effects are increased employment in the metallic industry and the benefits to national security from an important commodity --- aluminum is widely used in infrastructure --- that may reduce political uncertainties during the Ukraine-Russia conflict. The additional government revenue will also be beneficial in achieving domestic supply-side development through subsidies in periods of high inflation caused by supply-side shocks (citation), particularly in the energy and commodity sector. Furthermore, while the British tariff does create various DWL for society, it may be justified through the massive alleged subsides of Chinese producers in that the decreased consumption to $Q_{d2}$ prevented overconsumption and overproduction of aluminum extrusions. This retaliation may ensure a fairer international interdependence and increase the sustainability of production in the long term.
% Point 1, Job protection and fair competition long run. Subsides of misallocation in resources offset.

% Point 2, demand elasticity stuff, MLC and inflation.
\begin{figure}[H]
    \centering
    \tikzfig{assets/eco.txt}
    \caption{}
    \label{fig:eco}
\end{figure}
However, the elasticity of aluminum extrusion demand and supply can reduce the effectiveness of the policy that runs risk of inflating the already inflated economy in the short-run. Due to a lack of substitutes (citation), the short time period, and the importance of the extrusions in infrastructure projects, the British demand for these goods will be highly inelastic (figure \ref{fig:tariff}); the inflexibility of infrastructure required for extrusion productions in the short-run also reduces supply elasticity. So it requires a large tax to reduce imports and increase domestic production, implying a larger rise in consumer price of aluminum. It is likely that the policy will not only affect construction firms through higher costs, the interdependence between sectors of economy can propagate the higher cost of production within the economy due to the high utility of aluminum for infrastructure. The upwards shift in the $SRAS$ to $SRAS'$ in figure \ref{fig:eco} will decrease economic activity to $Y_2$ and increase price level to $PL_2$, creating cost-push inflation. Such high rates of inflation will generate uncertainty and reduce the real value of savings, affecting investors and savers.



\begin{figure}[H]
    \centering
    \tikzfig{assets/forex.txt}
    \caption{}
    \label{fig:forex}
\end{figure}

% Point 3, Tariff's consequences of retailiation and inflationary
Moreover, this inflationary pressure may be worsened if China takes retaliation actions in tit-for-tat for the loss of Chinese revenue through imposing UK machinery export tariffs. UK trade deficit will increase in such scenarios as its exports becomes more expensive in Chinese markets. This causes the pound to depreciate from $P_1$ to $P_2$ on the Forex market (figure \ref{fig:forex}) as the demand for British currency decreases to $AD'$. The fall in the purchasing power of the pound would increase the economy's cost of imports, especially against the large portion of Chinese commodities (citation), which fuels existing cost-push inflation (figure \ref{fig:eco}). A full on trade-war will benefit neither economy in the long-run, as demonstrated by the US (citation).

% Summary
% Affects stakeholders, priorities in long run, yet faces possible consequences
While the tariff on Chinese aluminum extrusions would benefit British producers and helping with supply-side growth in the long-term, the inelastic demand and supply of the product in an interdependent economy may increase short-run inflationary pressure --- worsened by the mere prospect of foreign retaliation.




\end{document}
