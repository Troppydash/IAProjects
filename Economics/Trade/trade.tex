\documentclass[a4paper,12pt]{article}
\usepackage[utf8]{inputenc}
\usepackage{graphicx}
\usepackage{amsmath}
\usepackage{float}
\usepackage{parskip}
\usepackage{varwidth}

% TIKZ
\usepackage{physics}
\usepackage{amsmath}
\usepackage{tikz}
\usepackage{mathdots}
\usepackage{yhmath}
\usepackage{cancel}
\usepackage{color}
\usepackage{siunitx}
\usepackage{array}
\usepackage{multirow}
\usepackage{amssymb}
\usepackage{gensymb}
\usepackage{tabularx}
\usepackage{extarrows}
\usepackage{booktabs}
\usetikzlibrary{fadings}
\usetikzlibrary{patterns}
\usetikzlibrary{shadows.blur}
\usetikzlibrary{shapes}

% TIKZ Figure
%\newcommand{\sized}[2][0.7\textwidth]{\begin{varwidth}{#1}#2\end{varwidth}}
\newcommand{\tikzfig}[1]{\input{#1}}

\linespread{1.2}

% TEXCOUNT
\usepackage{moreverb} % for verbatim ouput
\newcommand{\initTexcount}[1]{\immediate\write18{texcount -inc -incbib
		-sum #1 > .wordcount.tex}}
\newcommand\wordcount{
		\verbatiminput{.wordcount.tex}}

\usepackage[style=apa]{biblatex}
\addbibresource{trade.bib}
\usepackage{hyperref}


\title{The impacts on efficiency given UK's import tariffs on Chinese exported Aluminum Extrusions}
\author{\vspace{-8ex}}
\date{\vspace{-8ex}}

\begin{document}

\maketitle
Word count: 800\\
Article: https://www.reuters.com/markets/commodities/uk-may-impose-duties-up-29-chinese-aluminium-extrusions-2022-05-20/, ``UK may impose duties of up to 29\% on Chinese aluminium extrusions'', published by Reuters, written on 21/05/2022, accessed 07/06/2022\\
Topic: The Global Economy\\
Key Concept: Efficiency

%TC:ignore
%\initTexcount{trade.tex}
%\wordcount
%TC:endignore

\newpage
\section*{Article}
LONDON, May 20 (Reuters) - Britain may impose anti-dumping duties of up to 29\% on aluminium extrusions from China to protect domestic producers, a trade agency said on Friday.

Aluminium extrusions - widely used in the transport, construction and electronics industries - are being dumped in Britain at lower prices than they are sold in China, the Trade Remedies Authority said in a interim report.

"The TRA determined that there is already damage to the UK industry, having found clear evidence of price undercutting, indicating that UK businesses are struggling to compete with the dumped imports," a statement said.

Provisional measures will be imposed as the TRA completes its investigation, requiring Chinese companies exporting to Britain to provide a bank guarantee beginning on May 28, it added.

Duties ranging from 7.3\% to 29.1\% were recommended, depending on the company and the level of dumping margin, the interim report said.

Three companies were named - Press Metal International Group, Shandong Nanshan (600219.SS) and Haomei Group - plus there were categories for other co-operating and non-cooperating exporters.

Press Metal International is a Chinese unit of Malaysia's Press Metal (PMET.KL).

The TRA was established after Britain left the European Union to investigate unfair trade practices and measures.

The aluminium extrusion investigation is the first one resulting from a British industry claiming unfair trade practices, the TRA said.

One major firm that produces aluminium extrusions in Britain is Norway's Norsk Hydro (NHY.OL)

\newpage
% STRUCTURE

% DEFINITION
The article describes Britain's possible imposition of tariffs upon imported Chinese aluminum-extrusion as anti-dumping measures. Import tariffs are taxes imposed per unit good or service imported from a particular country --- a protectionist policy aiming to reduce international trade and assist domestic firms through reduced competition. The argument is suspected dumping: the exporting of goods and services under their costs of production often caused by governmental subsidies, oversupplying domestic market with competitive prices viewed as ``unfair trade practices''. I will be evaluating the policy's effectiveness and the economic consequences for the two nations through the key concept of efficiency.

% State the addition of tariff, brief connection to intended purpose.
The import tariff is likely to benefit British aluminum-extrusions producers in the short-run at the cost of allocative efficiency. A trade supply-demand diagram (figure \ref{fig:tariff}) can be used to explain this.

\begin{figure}[H]
	\centering
	 \tikzfig{assets/tariffs.txt}
	\caption{}
	\label{fig:tariff}
\end{figure}

The imposition of a proportional tariff on Chinese aluminum-extrusions will increase their COP at all supplied quantities, causing an upwards shift from $S(China)$ to $S'(China)$. Assuming that the Chinese supply is perfectly elastic for its large/efficient aluminum manufacturing sector \parencite{ilzetzki_2022}, the shift will increase market price from $P_w$ to $P_2$ as consumption decreases from $Q_{d1}$ to $Q_{d2}$.
%If the new price $P_2$ is below the equilibrium price where $D=S(Domestic)$
Imports will drop from $Q_{d1}-Q_{1}$ to $Q_{d2}-Q_{2}$ decreasing Chinese-firm revenue from $H+I+J$ to $I$; British producers will increase production from $Q_1$ to $Q_2$ increasing their revenue from $F+G$ to $A+B+F+G+H$. The British government will collect revenue $C$ due to the tariff. Overall, inefficiency of British producers and the decreased consumption both reduce welfare of size $B$ and $E$, a dead-weight loss on social surplus.

% Key concept
The key concept of efficiency is reflected in the reduced market efficiency and social surplus from decreased Britain-China trade caused by higher trade-barriers. While the protectionist measure may increase domestic firm efficiency in the long-run through a larger market with economies-of-scale, the inflationary pressure and prospects of Chinese retaliation can reduce investments and consumption that prevents the economies in capitalizing their comparative advantages.
%The key concept of interdependence is demonstrated in the interconnectivity in trade between the Chinese and British economy, including the interdependence between foreign producers and domestic consumers. The protectionist policy's reduction in social welfare and efficiency (of $B+E$) of comparative advantages between the nations is more apparent in this globally interdependent economy.

% TODO: Add this into the first point
%\begin{figure}[H]
%	\centering
%	\tikzfig{assets/eos.txt}
%	\caption{}
%	\label{fig:eos}
%\end{figure}

% CONCLUSION, LONG,SHORT-TERM, ASSUMPTIONS, STAKEHOLDERS, PRO CON, PRIORITiES
% Point 1, tariff's effect on domestic firms and foreign firms.
Positively, the tariff can reduce foreign market inefficiencies and promote productive production domestically through competitions and economies-of-scale. The immediate higher COP to Chinese firms can reduce the market inefficiency caused by subsidies in reducing the overconsumption and overproduction of aluminum. To compete domestically, British firms may spend the extra revenue in research \& development that can increase productivity and efficiency; the larger domestic quantity-demanded also encourages increased capital usage, leading to decreased average costs that creates economies-of-scale, increasing efficiency in production.
%More domestic effects are increased employment in the metallic industry and the benefits to national security from an important commodity --- aluminum is widely used in infrastructure --- that may reduce political uncertainties during the Ukraine-Russia conflict.
Increased government revenue from tariffs can also achieve domestic supply-side development through governmental R\&D in inefficient sectors --- subsidizing renewable energy, for example. While the subsidy may create market inefficiency in the short-run, the environmental benefits and the reduced negative externalities of production that renewable energy solves benefits the market efficiency in the long-term. This retaliation follows the infant-sector argument employed by China in technology that through fairer competitions, domestic efficient firms can ultimately create a comparative advantage.

\begin{figure}[H]
    \centering
    \tikzfig{assets/re.txt}
    \caption{}
    \label{fig:forex}
\end{figure}

However, the efficiency in nations' comparative advantages in production can be demolished if China takes retaliatory actions as tit-for-tat for the tariff through imposition of UK machinery export tariffs. The measure will shift the perfectly elastic supply of British exported machinery upwards in the Chinese capital market (figure \ref{fig:forex}), increasing market price from $P_w$ to $P_2$ and reducing Chinese consumption from $Q_{d1}$ to $Q_{d2}$. The inefficiency $E$ caused by decreasing consumption and $B$ caused by Chinese producers' relatively inefficiency marks the DWL in social surplus from the retaliation. Moreover, the decreased British imports from $Q_{d1}-Q_1$ to $Q_{d2}-Q_2$ will reduce British export quantities and receipts. The smaller output harms economies-of-scale of the British firms and increasing production inefficiencies reduces comparative advantages. A full-on trade-war that removes British and Chinese comparative advantages in machinery and aluminum will benefit neither economies in the long-run for under-consumption, demonstrated by the US-China trade-war \parencite{hass_denmark_2022}.


\begin{figure}[H]
    \centering
    \tikzfig{assets/eco.txt}
    \caption{}
    \label{fig:eco}
\end{figure}
Furthermore, in the medium-term, the low elasticity of aluminum-extrusion in demand and supply can risk inflation that reduces economic activity and efficiency. Due to a lack of substitutes \parencite{statista} and importance of extrusions in infrastructure projects, the British demand for these goods will be highly inelastic; the inflexibility of infrastructure required for extrusion productions also reduces supply elasticity. Therefore it requires a large tax to accomplish the desired reduction in imports (article stated 29\%), implying a large rise in consumer price of aluminum. The high utility of aluminum in infrastructure will propagate the higher COP within the British economy, increasing the $SRAS$ to $SRAS'$ as in figure \ref{fig:eco},
%It is likely that the policy will not only affect construction firms through higher costs, the interdependence between sectors of economy can propagate the higher cost of production within the economy due to the high utility of aluminum for infrastructure.
%The upwards shift in the $SRAS$ to $SRAS'$ in figure \ref{fig:eco} will
increasing price level from $PL_1$ to $PL_2$ and decreasing economic activity from $Y_1$ to $Y_2$.
% talk about how prospects of inflation can damage efficiency
The uncertainty of inflation in reduced real incomes and central bank interventions can reduce the propensity of consumption and investment through prospects of an eventual recession with lower economic activity increasing savings. In combination with prospects of retaliations, this risks under-consumption and reduces supply-side growth in the medium-term that harms allocative and production efficiency.
%Such high rates of inflation will generate uncertainty and reduce the real value of savings, affecting investors and saving-consumers.



% discuss DWL in new trade diagrams, and implications for efficiency for a full-on tradewar.

%Furthermore, this inflationary pressure may be worsened if China takes retaliation actions in tit-for-tat for the loss of Chinese revenue through imposing UK machinery export tariffs. UK trade deficit will increase in such scenarios as its exports becomes more expensive in Chinese markets. This causes the pound to depreciate from $P_1$ to $P_2$ on the Forex market (figure \ref{fig:forex}) as the demand for British currency decreases to $AD'$. The fall in the purchasing power of the pound would increase the economy's cost of imports, especially against the large portion of Chinese commodities \parencite{worldbank_2019}, which fuels existing cost-push inflation (figure \ref{fig:eco}). A full on trade-war will benefit neither economy in the long-run, as demonstrated by the US \parencite{hass_denmark_2022}.

% Summary
% Affects stakeholders, priorities in long run, yet faces possible consequences

While the tariff on Chinese aluminum-extrusions can increase efficiency of British producers through economies-of-scale, the inelasticity with high inflationary prospects and potential trade-wars will decrease investments and trade that reduces supply-side development \& allocative efficiency.

%the inelastic demand and supply of the product in an interdependent economy may increase short-run inflationary pressure --- worsened by the mere prospect of foreign retaliation.


% citations
% https://voxeu.org/article/surging-inflation-uk
\newpage
\printbibliography


\end{document}
