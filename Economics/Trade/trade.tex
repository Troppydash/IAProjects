\documentclass[a4paper,12pt]{article}
\usepackage[utf8]{inputenc}
\usepackage{graphicx}
\usepackage{amsmath}
\usepackage{float}
\usepackage{parskip}

\linespread{1.2}

\title{The possibilities of UK import tariffs}
\author{Terry Qi}

\begin{document}

\maketitle
Word count: 800\\
Article: https://www.reuters.com/markets/commodities/uk-may-impose-duties-up-29-chinese-aluminium-extrusions-2022-05-20/, ``UK may impose duties of up to 29\% on Chinese aluminium extrusions'', published by Reuters, written on 21/05/2022, accessed 07/06/2022\\
Topic: The Global Economy\\
Key Concept: Interdependence

\newpage

\section*{Article}
``UK may impose duties of up to 29\% on Chinese aluminium extrusions''

LONDON, May 20 (Reuters) - Britain may impose anti-dumping duties of up to 29\% on aluminium extrusions from China to protect domestic producers, a trade agency said on Friday.

Aluminium extrusions - widely used in the transport, construction and electronics industries - are being dumped in Britain at lower prices than they are sold in China, the Trade Remedies Authority said in a interim report.

"The TRA determined that there is already damage to the UK industry, having found clear evidence of price undercutting, indicating that UK businesses are struggling to compete with the dumped imports," a statement said.

Provisional measures will be imposed as the TRA completes its investigation, requiring Chinese companies exporting to Britain to provide a bank guarantee beginning on May 28, it added.

Duties ranging from 7.3\% to 29.1\% were recommended, depending on the company and the level of dumping margin, the interim report said.

Three companies were named - Press Metal International Group, Shandong Nanshan (600219.SS) and Haomei Group - plus there were categories for other co-operating and non-cooperating exporters.

Press Metal International is a Chinese unit of Malaysia's Press Metal (PMET.KL).

The TRA was established after Britain left the European Union to investigate unfair trade practices and measures.

The aluminium extrusion investigation is the first one resulting from a British industry claiming unfair trade practices, the TRA said.

One major firm that produces aluminium extrusions in Britain is Norway's Norsk Hydro (NHY.OL)

\newpage
\section*{Commentary}

% STRUCTURE

% DEFINITION
The article described the motifs behind Britain's possible imposition of imported Chinese aluminum extrusion as anti-dumping measures. Import tariffs are taxes imposed per unit good or service import from a particular country --- the proposed tariff is proportional of maximum 29\%. Such tariff imposition is a protectionist policy aiming to reduce international interdependence and assist domestic firms through reduced competition. The argument is based on dumping, the exporting of goods and services under their costs of production often caused by governmental subsidies, that was viewed by the TRA (Trade Rememdies Authority) as ``unfair trade practices''. I will be evaluating the policy's effectiveness and its potential consequences within the two nations.

% State the addition of tariff, brief connection to intended purpose.

% Key concept

% TODO: NEED THREE POINTS
% Point 1, tariff's effect on domestic firms and foreign firms.

% Point 2, Tariff's consequences of retailiation and inflationary. Compariative advantage of internationalization.

% Point 3, Job protection and fair competition long run. Subsides of misallocation in resources offset.

% Summary
% Affects stakeholders, priorities in long run, yet faces possible consequences reduce overall efficiency.



\end{document}
