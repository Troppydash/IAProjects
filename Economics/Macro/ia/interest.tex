\documentclass[a4paper,12pt]{article}
\usepackage[utf8]{inputenc}
\usepackage{graphicx}
\usepackage{amsmath}
\usepackage{float}
\usepackage{parskip}

\title{New Zealand increasing interest rates to curb inflationary pressure}
\author{Terry Qi}

\begin{document}

\maketitle

Word count: 791\\
Article: https://www.rnz.co.nz/news/business/453021/reserve-bank-announces-first-cash-rate-rise-in-seven-years, ``Reserve Bank announces first cash rate rise in seven years'', published by RNZ, written on 06/10/2021, accessed 19/11/2021\\
Topic: Macroeconomics\\
Key Concept: Intervention

\newpage

\section*{Article}

\textit{Reserve Bank announces first cash rate rise in seven years}

The Reserve Bank (RBNZ) has ignored the Covid-19 outbreak and raised the official cash rate (OCR) for the first time in seven years to head off growing inflation pressures.

It raised the OCR by a quarter of a percentage point to 0.5 percent, as expected, because of strongly rising prices, a hot housing market, and tight labour market.

The RBNZ's Monetary Policy Committee said the rise was justified even though the economy was likely to take a sharp hit from the current outbreak and lockdowns.

"The current COVID-19-related restrictions have not materially changed the medium-term outlook for inflation and employment since the August Statement."

It said household and business activity has been strong going into the shutdown and it was expected to rebound as it had previously.

"Ongoing fiscal policy support, and a strong terms of trade provide confidence that economic activity will recover quickly as alert level restrictions ease. Recent economic indicators support this picture," the committee said in a statement.

The RBNZ had been expected to raise rates in August, but decided at the last minute to keep rates unchanged because of the Covid uncertainty.

It acknowledged some businesses would be "badly affected" by the lockdowns but decided it needed to act to meet its target of maximum employment and inflation anchored around 2 percent.

"There will be longer-term implications for economic activity both domestically and internationally from the pandemic," adding the way to lessen the impact and disruptions was vaccination.

It also signalled further rate rises are coming.

"Further removal of monetary policy stimulus is expected over time, with future moves contingent on the medium-term outlook for inflation and employment."

Kiwibank chief economist Jarrod Kerr said it was clear that the central bank felt it could wait no longer.

"The Kiwi economy has solid momentum and the RBNZ has good reason to withdraw stimulus.

"And the RBNZ won't stop here. October marks the beginning of a new chapter for the cash rate: Onwards and upwards."

He expected a similar sized rate rise in November, February, and May, when the RBNZ would have a pause.

Retail banks had a mixed reaction to the decision.

ASB gave a commitment not to change rates this year, ANZ said it would pass on about half of the OCR by lifting loan rates by 15 basis points, while Kiwibank passed on the full amount to its floating mortgages and some of its deposit rates.

All 20 economists polled by Reuters ahead of the announcement believed New Zealand's official cash rate would be moved upwards from its record low 0.25 percent, to 0.5 percent.

The move has been seen as necessary to cool inflation, which is running at its fastest growth rate in more than a decade, unemployment is down to 4 percent, and soaring house prices also factor into the decision.

National Party leader Judith Collins says the Reserve Bank has responded to "ill-conceived" government spending, which has created inflationary measures.

"The longer these lockdowns continue, the more money's going to have to be pumped in by the government into the economy, and the borrowing's just going up by astronomical amounts," she said.

The government was the "biggest influencer on inflation", Collins said, and "New Zealanders who are currently having difficulty paying their mortgage or their costs, they're going to find that it just becomes harder".

Expect more pain in the long run
John Bolton from mortgage broker Squirrel said lenders have been pricing in the Reserve Bank increase for three or four months.

"If I use the one year fixed rates as an example, I mean that got as low as about 2.19 percent. And today, those are already up at around 2.79 percent. So, we've already sort of seen over half a percent increase in those fixed rates and that's been in anticipation of the Reserve Bank moving rates."

Bolton said in the short-term, fixed mortgage rates are unlikely to move too much, but floating rates are a different story as it's more linked to the OCR.

He said while it's a small, well-sign posted rise, people paying off mortgages are entering a new environment.

Michael Gallagher from Financial Advice Hawkes Bay said the worry is in the long term.

"The bigger concern is the talk of rates obviously continuing to rise and then it becomes a much bigger cost for people in the longer term. So it is a real concern going forward and how quickly these rates do rise."

\newpage
\section*{Topic}
\section*{Theory}

\section*{Evaluation}

\section*{Summary}

\end{document}
