\documentclass[a4paper,12pt]{article}
\usepackage[utf8]{inputenc}
\usepackage{graphicx}
\usepackage{amsmath}
\usepackage{float}
\usepackage{parskip}

\begin{document}

\section{Evaluation of the consequences of economic growth}

% conclusion, length of time, assumptions, stakeholders, priorities, pros & cons

Economic growth is defined as the increase in the real GDP of an economy, but the different forms of long run growth will are evaluated differently. 

Economic growth through the means of increasing the macroeconomic equilibrium position decreases the recessionary gap. This is often caused by the increase in AD, through one or more of its components; or visualized as a point within the PPF curve moving outwards. The main benefit of such forms of growth is the likely decrease in unemployment rates from the shortening of the recessionary gap. Higher employments will decrease the social and economic costs of the unemployed in the forms of higher crime rates and unemployment benefits --- both requires government spending and presents an opportunity cost in public sector spending. Additionally, living standards is likely to increase in the short run, for households income will increase on average, and through the interdependence between households and firms, increases investment funding, bringing technological improvements.

However, assuming the economy is close to the level of full employment, such economic growth may bring high inflationary pressure in the long run. For the AS is perfectly inelastic at the level of full employment, any increase in AD can only bring a rise in the price level. This negates the perceived income increases of households, and provides decreasing consumer expectation, decreasing investments and have the chance of spiraling with wages.

Economic growth through the means of a rightwards shift of the aggregate supply curve may increase the recessionary gap in the short run. But it is assumed that with positive population growth, AD will naturally increase overtime, creating inflationary pressure if AS remain static, decreasing unemployment caused by the shift of the AS in the long run. Therefore AS increases allows for non-inflationary growth of the economy, increasing the average living standards via a increase in the quality and quantity of factors of productions and thus the products. Additionally, if GDP were to increase, then the average income must too. Non-inflationary income increases creates greater tax revenue for the government, which can be spent on infrastructure and education, further increasing economic growth in the short and long run.

But the impact of the growth may not benefit all households evenly. It is possible for the economic growth be caused and benefited by only the wealthy, as indicated by the increase in the GINI coefficient of China (citation) --- through the 20 years of high economic growth. Additionally, the increased AS may imply a longer working hours and worsen working conditions as to save costs, further increasing inequality in the economy and may even decrease the living standards. Plus, increases in income may not increase the perceived living standard, for people will never satisfied by their wants. This presents some potentially beneficial consequences of economic growth.

\end{document}
