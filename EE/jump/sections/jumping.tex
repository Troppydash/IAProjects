\section{Jumping motion}
% main is with straight jumping
I will consider and evaluate multiple models to maximize the displacement jumped, this will include the modeling of the player motions as an differential equation assuming continuous time, and combine the engine mechanics into a singular discrete equation.

\subsection{Equations of motion}
First let us construct the differential equations that govern the player motion, maybe the answer is as simple as an optimization problem. Let $\ta$, $\tv$, and $\tp$ be the two dimensional acceleration, velocity, and position. It is assumed that player can perfectly control $\ta$ at all time $t \in [0, t_f]$ via their keyboard and mouse --- the time period $t\in[0,t_f]$ will be assumed for all functions in this section.

Initially consider the case without any maximum velocity restrictions, the acceleration is:
\begin{align*}
    \ta = w\tunit{a} = w\tpar{x(t)}{y(t)},
\end{align*}
where $x(t)$ and $y(t)$ are functions of time and is a unit vector where
\[
x^2 + y^2 = 1.
\]

The value of $w$ is different under continuous assumptions, albeit still reminding constant throughout. We will define $w$ by the maximized acceleration of the player divided by the change in time, that is: $w = \frac{\gamma_1}{\tau}$. The justification is that the discrete scalar $w$ can be considered as a tiny change in velocity, which can be rearranged (notice that this uses equation \ref{eq:dis_vel} where $w=\gamma_1$):
\begin{align*}
    d\tv &=  \tv' - \tv\\
    &= (\tv + w \tunit{a}) - \tv\\
    &= w \tunit{a}\\
    \text{therefore:}&\\
    \frac{d\tv}{\tau} &= \frac{w}{\tau} \tunit{a},
\end{align*}
and as $\tau \to 0$, the equation will approach the time derivative of the velocity (acceleration) of the player, where the constant scalar $w=\frac{\gamma_1}{\tau} = LA$. This will result in the  differential equation in figure

\begin{figure}[H]
 \centering
 \[
    \frac{d\tv}{dt} = \frac{w}{\tau} \tunit{a}
 \]
 \caption{The unconstrainted differential equation of motion}
 \label{eq:un_diff}
\end{figure}

Now notice that the velocity at any point is the time integral of the acceleration, which can be expanded into the x,y components like so:
\begin{align*}
    \tv &= \int w\tunit{a} \,dt + c\\
    &= \tpar{\int w x(t) \,dt}{ \int w y(t) \,dt} + \tpar{c_x}{c_y}\\
    &= w\tpar{X(t)}{Y(t)} + \tpar{c_x}{c_y},
\end{align*}
where $X(t)$ and $Y(t)$ are the time integrals of $x(t)$ and $y(t)$.

The constants ($c_0$,$c_1$) can be found by plugging in $t=0$, for we can define the constants in terms of the initial velocities:
\begin{align*}
    v_0 &= w \tpar{X(0)}{Y(0)} + \tpar{c_x}{c_y}\\
    c_x &= v_{0x} -wX(0)\\
    c_y &= v_{0y} -wY(0).
\end{align*}

The displacement $\tp$ is the time integral of the velocity:
\begin{align*}
    \tp &= \int \tv \, dt + k\\
    &= w \tpar{\int X(t) \, dt}{\int Y(t) \, dt} + \tpar{tc_x}{tc_y} + \tpar{k_x}{k_y}\\
    &= w \tpar {\tfx(t)}{\tfy(t)}  + t\tpar{c_x}{c_y}  + \tpar{k_x}{k_y},
\end{align*}
where $\tfx$ and $\tfy$ are the time integral of $X$ and $Y$.

The constants $k_x$ and $k_y$ can be found by evaluating at $t=0$:
\begin{align*}
    \tp_0 &= w \tpar{\tfx(0)}{\tfy(0)} + \tpar{k_x}{k_y} = \tpar{0}{0}\\
    k_x &= -w\tfx(0)\\
    k_y &= -w\tfy(0).
\end{align*}

Therefore, the position of the player, when $L >> 0$, at all time $t$ in the respective domain is show in figure \ref{eq:jumping} (notice that $\tfx'(t) = X(t)$ and $\tfy'(t) = Y(t)$):

\begin{figure}[H]
    \centering
    \begin{align*}
        \tp(t)_x &= w\tfx(t) + t(v_{0x} - w\tfx'(0)) - w\tfx(0)\\
        \tp(t)_y &= w\tfy(t) + t(v_{0y} - w\tfy'(0)) - w\tfy(0).
    \end{align*}
    \caption{The Jumping Motion equations}
    \label{eq:jumping}
\end{figure}


\subsection{Restrictions}
Step 3 states that the velocity is limited through the projection of the current velocity onto the player acceleration. I thought to define the restrictions so that maximum acceleration magnitudes are achieved at all time.

The maximum magnitudes of velocity after every frame is achieved when $L- v \cdot \tunit{a} \ge LA\tau$, for then $\gamma_2 \ge \gamma_1$ and $w = \gamma_1$, for this ensures that acceleration is at a maximum. Additionally, we can see that both the $x$ and $y$ position are positively correlated to $w$, meaning that its maximization will result in higher overall displacement. However, I wonder if this is accurate for the new velocity $\tv'$ is the sum of current velocity with the acceleration, the angle between $\tunit{a}$ and $\tv$ dictates the magnitude of the updated velocity --- could it sometimes be worth it to forgo maximum acceleration to decrease the angle, resulting in a higher updated velocity?

For now, consider the restriction so that acceleration will be maximized at all $t$ in the time domain:
\begin{align*}
    L - v \cdot \tunit{a} &\ge LA\tau\\
    L - (v_x \tunit{a}_x + v_y \tunit{a}_y) &\ge LA\tau.
\end{align*}

Notice that $\tunit{a}_x = x(t)$ and $\tunit{a}_y = y(t)$ and:
\begin{align*}
    v_x &= wX(t) + c_x\\
    v_y &= wY(t) + c_y,
\end{align*}
therefore:
\begin{align*}
    LA\tau\ &\le L - x(t)(wX(t)+c_x) - y(t)(wY(t)+c_y)\\
    L - LA\tau &\ge x(t)(wX(t)+c_x) + y(t)(wY(t)+c_y)\\
    L - LA\tau &\ge x(t)(wX(t) + v_{0x} -wX(0)) + y(t)(w(Y) + v_{0y} -wY(0))\\
    \frac{L - LA\tau}{w} &\ge X'(t)(X(t) + \frac{v_{0x}}{w} - X(0)) + Y'(t)(Y(t) + \frac{v_{0y}}{w} -Y(0)),
\end{align*}
because $w=\gamma_1=LA\tau$, and $x(t) = X'(t), \quad y(t) = Y'(t)$, the simplified constraint of the jumping motion equations is displayed in figure \ref{eq:constraint}.
\begin{figure}[H]
    \centering
    \[
        \frac{1}{A\tau} - 1 \ge X'(t)(X(t) + \frac{v_{0x}}{w} - X(0)) + Y'(t)(Y(t) + \frac{v_{0y}}{w} -Y(0))
    \]
    \caption{The Constraint Equation}
    \label{eq:constraint}
\end{figure}

This restriction shows that for acceleration each frame to be maximized, the RHS must be less than the LHS constant for all $t \in [0, t_f]$.

\subsection{Motion with restrictions}
To simulate the continuous motion described here, one can take advantages of the absolute function and its derivatives and integrals, along with computers to simulate differential equations using Euler's approximation.

Although the acceleration constant $w$ is defined piecewise on $\gamma_1$ and $\gamma_2$ at each frame, of which are functions of velocity $\tv$, I've devised a method to encode it using standard arithmetic operations and an additional absolute $|x|$ operator only. The purpose of this approach is that the absolute function is differentiable and integratable --- perhaps we can write our restrictive jumping motion as a single differential equation and solve it?

Firstly we need to derive the required fundamental functions. The absolute operator can be defined as a piecewise function itself, the absolute value of $0$ will be ignored here as my method does not encode the scenario when $x=0$ (though this can be fixed during computation by adding a tiny epilson of $0.00001$ to $x$):
\[
    |x| = \begin{cases}
        x & x > 0\\
        -x & x < 0
    \end{cases}.
\]

We can define a sign function that is $+1$ when the value is positive, and $-1$ when it is negative:
\[
    \text{sign}(x) = \frac{|x|}{x} = \begin{cases}
        +1 & x > 0\\
        -1 & x < 0
    \end{cases},
\]
and define a positive function that is similar to the sign function, but returns $0$ when it is negative:
\[
    \text{pos}(x) = \frac{\text{sign}(x) + 1}{2} = \begin{cases}
        +1 & x > 0\\
        0 & x < 0
    \end{cases}.
\]

We also need a minimum function, because scalar $w$ is the minimum of $\gamma_1$ and $\gamma_2$ when $\gamma_2$ is positive (Step 3, \autoref{sec:source}, \nameref{sec:source}) --- this function is also not defined if $a$ equals $b$,
\[
    \text{min}(a, b) = b \text{pos}(a - b) + a \text{pos}(b - a) = \begin{cases}
        a & b > a\\
        b & a > b
    \end{cases}.
\]

Now consider the piecewise scalar $w$, it is defined as
\[
    w = \begin{cases}
        \min(\gamma_1, \gamma_2) & \gamma_2 > 0\\
        0 & \gamma_2 \le 0
    \end{cases},
\]
we want to model a function that is zero when $\gamma_2$ is negative, and the minimum function otherwise. I've come up with this equation:
\[
    w = \text{pos}(\gamma_2)\min(\gamma_1, \gamma_2),
\]
notice that the $\text{pos}(\gamma_2)$ will be zero when $\gamma_2$ is negative, and $1$ otherwise. This is equivalent to the piecewise definition.

Furthermore, the equation can now be substituted by the definition of the minimum function:
\[
    w = \text{pos}(\gamma_2)(\gamma_2 \text{pos}(\gamma_1 - \gamma_2) + \gamma_1 \text{pos}(\gamma_2 - \gamma_1)).
\]

By replacing the variable $w$ in the differential equation of equation \ref{eq:un_diff} with the above equality, the restrictive jumping motion is now a first order differential equation (equation \ref{eq:reom}).

\begin{figure}[H]
    \centering
    \[
        \frac{d\tv}{dt} = \frac{\text{pos}(\gamma_2) (\gamma_2\text{pos}(\gamma_1 - \gamma_2) + \gamma_1 \text{pos}(\gamma_2 - \gamma_1))}{\tau} \tunit{a}
    \]
    \caption{The restricted equation of motion}
    \label{eq:reom}
\end{figure}

The first order differential equation can be simulated with Euler's method. Recall that for step size $h$, updating function $f(t_n, \tv_n) = \frac{d\tv}{dt}$, the velocity at the $n$th step can be computed recursively such that:
\[
    \tv_{n+1} = \tv_{n} + h f(t_n, \tv_n).
\]

Since the displacement is simply the integral of the velocity, we can represent the discrete values of $\tv_{n}$ with a summation weighted by the step size $h$. This will result in the restricted jumping equation in equation \ref{eq:rje}. I will be using this method to compute the continuous restricted jumping distance by approximating $h \to 0$, and simulating the source engine by setting $h = \tau$.

\begin{figure}[H]
    \centering
    \[
        \tp(t) = \lim_{h \to 0} \sum_{n=0}^{\ceil*{\frac{t}{h}}} v_{n} h
    \]
    \caption{The restricted jumping equation}
    \label{eq:rje}
\end{figure}


