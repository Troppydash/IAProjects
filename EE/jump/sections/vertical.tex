\section{Vertical motion}
Universal across the various metrics are the vertical motion of the player. Firstly notice that the z-axis motion is separated from the x,y-axis motion, meaning that it can be computed on its own.

Let $p$, $v$, $a$ be the z-axis position, velocity, and acceleration as a function of time $t$ and engine constants $S$, also define $p_0$, $v_0$ to be the initial conditions. Because of the velocity update in equation \ref{eq:gv}:
\[
\frac{\Delta v}{\tau} = g,
\]
because $\tau$ is very small, it can be assumed that:
\[
v' = g.
\]
Similarly, the position update can also be assumed to be continuous (equation \ref{eq:gp}):
\begin{align*}
    \frac{\Delta p}{\tau} &= v(t+\tau)\\
    p' &= v.
\end{align*}
Therefore it is possible to compute the airtime of the player upon a jump action. For the acceleration of gravity is constant, the kinematic equations may be used (citation).
\begin{align*}
    s &= s_0 + ut + \frac{1}{2} a t^2\\
    0 &= 0 + v_0 t + \frac{1}{2} g t^2\\
    \text{therefore:}&\\
    t &= \frac{-v_0 \pm \sqrt{v_0^2}}{g}\\
    &= \frac{-2v_0}{g} \,\, \text{or} \,\, 0.
\end{align*}
Because $t=0$ is the first frame when the player executes the jump action, the airtime is the first root of the quadratic, and therefore:
\begin{align*}
    t &= -2 \frac{v_0}{g}\\
    &= -2 \times \frac{280\pm 10}{-800}\\
    &= 0.70 \pm 0.03 \si{s}
\end{align*}

The result is very similar to the recorded airtime of $0.75\pm 0.02$ with their uncertainty overlapping, the assumption of a small $\tau$ may also contribute to the difference between the observed with the calculated time. This airtime will be denoted by the symbol $t_f$.

Therefore if the initial displacement at $t=0$ ($\tp(0)$) has a length of zero, the total displacement as a result of the jump is $\tmag{\tp(t_f)}$.
