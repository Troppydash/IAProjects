\documentclass[a4paper,11pt]{article}
\usepackage[a4paper,margin=1in,footskip=0.25in]{geometry}
\usepackage[utf8]{inputenc}

% science
\usepackage{amsmath}
\usepackage{array}
\usepackage{siunitx}

% code
% Default fixed font does not support bold face
\DeclareFixedFont{\ttb}{T1}{txtt}{bx}{n}{10} % for bold
\DeclareFixedFont{\ttm}{T1}{txtt}{m}{n}{10}  % for normal

% Custom colors
\usepackage{color}
\definecolor{deepblue}{rgb}{0,0,0.5}
\definecolor{deepred}{rgb}{0.6,0,0}
\definecolor{deepgreen}{rgb}{0,0.5,0}

\usepackage{listings}

% Python style for highlighting
\newcommand\pythonstyle{\lstset{
        language=Python,
        basicstyle=\ttm,
        morekeywords={self},              % Add keywords here
        keywordstyle=\ttb\color{deepblue},
        emph={MyClass,__init__},          % Custom highlighting
        emphstyle=\ttb\color{deepred},    % Custom highlighting style
        stringstyle=\color{deepgreen},
        frame=tb,                         % Any extra options here
        showstringspaces=false
}}

\newcommand\pythonstyleln{\lstset{
        language=Python,
        basicstyle=\ttm,
        morekeywords={self},              % Add keywords here
        keywordstyle=\ttb\color{deepblue},
        emph={MyClass,__init__},          % Custom highlighting
        emphstyle=\ttb\color{deepred},    % Custom highlighting style
        stringstyle=\color{deepgreen},
        frame=tb,                         % Any extra options here
        showstringspaces=false,
        numbers=left
}}

% Python environment
\lstnewenvironment{python}[1][]
{
    \pythonstyle
    \lstset{#1}
}
{}

\lstnewenvironment{pythonln}[1][]
{
    \pythonstyleln
    \lstset{#1}
}
{}

% Python for external files
\newcommand\pythonexternal[2][]{{
        \pythonstyle
        \lstinputlisting[#1]{#2}}}

% Python for inline
\newcommand\pythoninline[1]{{\pythonstyle\lstinline!#1!}}

% layout
\usepackage{float}
\usepackage{parskip}
\usepackage{graphicx}
\usepackage{circuitikz}
\usepackage{longtable}
\usepackage{hyperref}
\usepackage[export]{adjustbox}
\usepackage{hhline}
\usepackage{chngcntr}

% referencing
\usepackage[style=apa]{biblatex}
\addbibresource{jump.bib}

% table centering
\renewcommand{\arraystretch}{1.3}
\newcolumntype{P}[1]{>{\raggedright\arraybackslash}p{#1}}
\newcommand{\tptt}{$\times\,$}

% figures labelings
\usepackage{chngcntr}
\counterwithin{figure}{section}

\usepackage{caption}
\usepackage{subcaption}
\usepackage{wrapfig}

% fancy page numbers
\usepackage{fancyhdr}
% bottom right
\pagestyle{fancy}
\fancyhf{}
\fancyfoot[R]{Page \thepage}
\fancypagestyle{plain}{%
    \renewcommand{\headrulewidth}{0pt}%
    \fancyhf{}%
    \fancyfoot[R]{Page \thepage}%
}
%go
\renewcommand{\headrulewidth}{0pt}

\linespread{1.5}

\usepackage{hyperref}

\title{The Optimization of Air Movements within a Video Game Jump Action}

\author{Word Count: < 4000}
\date{}

\begin{document}

    \maketitle

    \newpage

    \tableofcontents

    \newpage

\section{Introduction}

\paragraph{Research Question:}
\begin{quote}
    To what extent can a player travel in distance/displacement through a jump action in optimistic conditions of vanilla settings, within the games made with the source engine?
\end{quote}

The problem originated my personal curiosity within the limitations of the video game ``Counter Strike'' that occupies a large part of my life. It came about as I was getting into the community of counter-strike long jumping --- a community aiming to exploit the video game's physics to achieve the longest jumping distance. Therefore I want to explore this optimization problem with various methods in an extended essay and to further improve my knowledge within this ``niche`` community by exploring the various forms of this question.

\subsection{Defining the problem}
The ``source'' engine is the base of plenty of popular videos games in terms of handling the players, their surroundings, and their respective physical interactions. It allows freedom in setting various engine constants that fit each video game's playstyle --- with the constants of ``Counter Strike - Source`` are used as a measure throughout this exploration. I will attempt to define the problem with analogy to real-life kinematic assumptions.

% from p0 to p1, and to finishing this essay










\nocite{*}
\newpage
\printbibliography

\end{document}
