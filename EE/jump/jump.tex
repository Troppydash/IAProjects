\documentclass[a4paper,12pt]{article}
\usepackage[utf8]{inputenc}
\usepackage[a4paper,margin=1in,footskip=0.25in]{geometry}

%tex
\usepackage{physics}
\usepackage{amsmath}
\usepackage{tikz}
\usepackage{mathdots}
\usepackage{yhmath}
\usepackage{cancel}
\usepackage{color}
\usepackage{siunitx}
\usepackage{array}
\usepackage{multirow}
\usepackage{amssymb}
\usepackage{gensymb}
\usepackage{tabularx}
\usepackage{extarrows}
\usepackage{booktabs}
\usetikzlibrary{fadings}
\usetikzlibrary{patterns}
\usetikzlibrary{shadows.blur}
\usetikzlibrary{shapes}



% science
\usepackage{amsmath}
\usepackage{array}
\usepackage{siunitx}
\usepackage{amssymb}

% code
% Default fixed font does not support bold face
\DeclareFixedFont{\ttb}{T1}{txtt}{bx}{n}{10} % for bold
\DeclareFixedFont{\ttm}{T1}{txtt}{m}{n}{10}  % for normal

% Custom colors
\usepackage{color}
\definecolor{deepblue}{rgb}{0,0,0.5}
\definecolor{deepred}{rgb}{0.6,0,0}
\definecolor{deepgreen}{rgb}{0,0.5,0}

\usepackage{listings}

% Python style for highlighting
\newcommand\pythonstyle{\lstset{
        language=Python,
        basicstyle=\ttm,
        morekeywords={self},              % Add keywords here
        keywordstyle=\ttb\color{deepblue},
        emph={MyClass,__init__},          % Custom highlighting
        emphstyle=\ttb\color{deepred},    % Custom highlighting style
        stringstyle=\color{deepgreen},
        frame=tb,                         % Any extra options here
        showstringspaces=false
}}

\newcommand\pythonstyleln{\lstset{
        language=Python,
        basicstyle=\ttfamily\footnotesize,
        morekeywords={self},              % Add keywords here
        keywordstyle=\ttb\color{deepblue},
        emph={MyClass,__init__},          % Custom highlighting
        emphstyle=\ttb\color{deepred},    % Custom highlighting style
        stringstyle=\color{deepgreen},
        frame=tb,                         % Any extra options here
        showstringspaces=false,
        numbers=left,
        breaklines=true
}}

% Python environment
\lstnewenvironment{python}[1][]
{
    \pythonstyle
    \lstset{#1}
}
{}

\lstnewenvironment{pythonln}[1][]
{
    \pythonstyleln
    \lstset{#1}
}
{}

% Python for external files
\newcommand\pythonexternal[2][]{{
        \pythonstyle
        \lstinputlisting[#1]{#2}}}

% Python for inline
\newcommand\pythoninline[1]{{\pythonstyle\lstinline!#1!}}

% layout
\usepackage{float}
\usepackage{parskip}
\usepackage{graphicx}
\usepackage{longtable}
\usepackage[export]{adjustbox}
\usepackage{hhline}
\usepackage{chngcntr}

% referencing
\usepackage[style=apa]{biblatex}
\addbibresource{jump.bib}

% figures labelings
\usepackage{chngcntr}
\counterwithin{figure}{section}
\numberwithin{equation}{section}

\usepackage{caption}
\usepackage{subcaption}
\usepackage{wrapfig}

% fancy page numbers
\usepackage{fancyhdr}
% bottom right
\pagestyle{fancy}
\fancyhf{}
\fancyfoot[R]{\thepage}
\fancypagestyle{plain}{%
    \renewcommand{\headrulewidth}{0pt}%
    \fancyhf{}%
    \fancyfoot[R]{\thepage}%
}
%go
\renewcommand{\headrulewidth}{0pt}

% extra maths commands
\usepackage{mathtools}
\DeclarePairedDelimiter\ceil{\lceil}{\rceil}
\DeclarePairedDelimiter\floor{\lfloor}{\rfloor}

\newcommand{\tvec}[1]{\boldsymbol{#1}}
\newcommand{\tunit}[1]{\boldsymbol{\hat{#1}}}
\newcommand{\tmag}[1]{\|#1\|}
\newcommand{\tang}[1]{\left\langle #1 \right\rangle}
\newcommand{\td}{\tvec{d}}
\newcommand{\ta}{\tvec{a}}
\newcommand{\tv}{\tvec{v}}
\newcommand{\tp}{\tvec{p}}
\renewcommand{\arraystretch}{1}
\newcommand{\tpar}[2]{\begin{bmatrix}#1\\#2\end{bmatrix}}
\newcommand{\tpars}[3]{\begin{bmatrix}#1\\#2\\#3\end{bmatrix}}
\newcommand{\tfx}{\mathcal{X}}
\newcommand{\tfy}{\mathcal{Y}}

\newcommand{\I}{{i\mkern1mu}}
\newcommand{\tzero}{\tvec{0}}

\newcommand{\sized}[2][0.4]{\setlength{\fboxrule}{0pt}\framebox[#1\textwidth]{#2}}

\linespread{1.6}

\usepackage[bottom]{footmisc}

\newcommand{\citefoot}[1]{\footnote{\citeall{#1}}}
\newcommand{\citeall}[1]{\citeauthor{#1}, \citetitle{#1} (\citeyear{#1})}


% change section title font
\usepackage{sectsty}
\sectionfont{\fontsize{12}{12}\selectfont}
\subsectionfont{\fontsize{12}{12}\selectfont}
\subsubsectionfont{\fontsize{12}{12}\selectfont}

\usepackage{hyperref}



% JUPYTER NOTEBOOK



\title{The Optimization of Jump Distances within a Video Game}

\author{Word Count: 3998}
\date{Subject: Mathematics}

%\includeonly{
    %sections/jumping.tex,
 %   sections/models.tex
%}
\begin{document}

    \maketitle

    \paragraph{Research Question:} How to maximize the player's x,y-axis displacement during one jump above a flat z-axis plane with achievable initial conditions in the video game ``Counter Strike, Source''?

    \newpage
    % texcount

    \newpage

    \tableofcontents

    \newpage



% introduction
%  research question
%  reasons
%  definitions

% mechanics
%  vectors and quirk dot product
%  complex numbers and differential equation

% Z-axis modelling
%  continous updating
%  discrete updating
%  evaluation

% XY-axis modelling
%  continous straightline
%  continous right turning at fixed rate
%  continous euler-lagrange
%  discrete velocity maximizing
%  discrete velocity maximizing with let go
%  discrete displacement maximizing
%  result, significance

% Conclusion


% first model and my best attempt



\section{Introduction}

\paragraph{Research Question:}
\begin{quote}
    To what extent can a player travel in distance/displacement through a jump action in optimistic conditions of vanilla settings, within the games made with the source engine?
\end{quote}

The problem originated my personal curiosity within the limitations of the video game ``Counter Strike'' that occupies a large part of my life. It came about as I was getting into the community of counter-strike long jumping --- a community aiming to exploit the video game's physics to achieve the longest jumping distance. Therefore I want to explore this optimization problem with various methods in an extended essay and to further improve my knowledge within this ``niche`` community by exploring the various forms of this question.

\subsection{Defining the problem}
The ``source'' engine (citation) is the base of plenty of popular videos games in terms of handling the players, their surroundings, and their respective physical interactions. It allows freedom in setting various engine constants that fit each video game's playstyle --- with the constants of ``Counter Strike - Source`` being used as a measure throughout this exploration. I will attempt to define the problem mathematically here.

% from p0 to p1, and to finishing this essay
\paragraph{Notations} Firstly, the notation used in the analysis are as follow:

All n-dimensional vectors are represented in bold styles like
\[
\tvec{v},
\]
their magnitudes, or length are denoted by two double vertical lines, with the components of a vector written as a subscript $x$, $y$, or $z$ like
\[
\tmag{\tvec{v}} = \sqrt{v_x^2 + v_y^2 + v_z^2}.
\]

The vector can also be expanded into its components, and is written with angled brackets or with parenthesis:
\[
\tvec{v} = \tang{v_x, v_y, v_z} = \tpars{v_x}{v_y}{v_z}.
\]

The unit vectors are vectors scaled down by their magnitudes, resulting in a magnitude of 1, and is written with a hat on top:
\[
\tunit{v} = \frac{\tvec{v}}{\tmag{v}}.
\]

The dot product between two vectors $\tvec{v}$ and $\tvec{w}$ is defined by
\[
\tvec{v} \cdot \tvec{w} = v_x w_x + v_y w_y + v_z w_z + \ldots,
\]
but their cross product will not be used as it only applies in the third dimension.

Due to the nature of the source engine, the problem may be represented and optimized differently because of the use of discrete time frames rather than continuous time. But experimentally the differences are small and insignificant --- this will be shown later. Both approaches will thus be used, sometimes interchangeably.

The problem can be formally defined as so:
\begin{quote}
    Given initial position $\tvec{p}(0)$ and velocity $\tvec{v}(0)$ as three dimensional vectors when the player initializes a jump event, and a list of engine constants $S = \{j_0, j_1, j_2, \ldots\}$. If $\ta(t)$ is the acceleration vector function representing the player acceleration at every time $t$, with $\tv(t, S)$ and $\tp(t)$ being the respective derived velocity and position of the player at time $t$, what is the maximum displacement $\tmag{\tp(t_f) - \tp(0)}$ achievable if $t_f$ is the time when the player lands from the jump?
\end{quote}

\subsection{Source mechanics}
While the problem may look simple in the context of kinematics and projectile motions, the ``quirks'' in the source engine will increase its complexity. Therefore it is necessary to explore the specifics of the physics engine.

The engine time unit is in seconds; the engine distance unit is in hammerunits (hu), which are to be implied during all following calculations. The player's position and velocities are represented by $\tp$ and $\tv$ in the engine, of which they are updated every frame by the player controllable acceleration $\ta$. Commonly, the source engine runs at an integer number of frames per second (citation), $64$ will be used from now. Let $\tau$ be the time between frames (also know as $\Delta t$) and it has a value of $\tau = \frac{1}{64} \approx 0.0156 \si{s}$. For every $\tau$ seconds after $t=0$, the velocity is updated as so (citations):
\begin{enumerate}
    \item Apply gravitational acceleration
    \item Apply friction onto velocity
    \item Calculate magnitude of movement acceleration
    \item Apply movement acceleration
\end{enumerate}

% took 45 frames, v0 = 284hu/s, 60fps
\paragraph{Step 1} Let $g$ be the gravitational acceleration on the $z$ axis, it is set by an engine constant \verb|sv_gravity| (citations) --- set to $800$ within the game in question. Using a method called leapfrog interpolation, and let $\tv'$, $\tp'$ be the new velocity, position next frame of frame time $\tau$, the operation is (assuming $g$ is negative)
\begin{align}
    \tv'_z &= \tv_z + g\tau \label{eq:gv}\\
    \tp'_z &= \tp_z + \tv_z\tau + \frac{1}{2}g\tau^2 \label{eq:gp}.
\end{align}

Because the updates reflect the kinematic equations of a projectile with constant acceleration, it can be treated with assumptions of continuous time and will be evaluated separate to the $x$ and $y$ motions.


\paragraph{Step 2} Let $\lambda(\tv)$ be the friction function. For the player is in the air, no friction will be applied and we can set $\lambda(\tv) = \tv$.

\paragraph{Step 3} The source engine limits the velocity through an engine constant \verb|sv_maxspeed| ($L$) set to $30$ in air. However, the speed is limited by the projection of the velocity vector $\tv$ onto the acceleration vector $\ta$, with the player having control of $\ta$ at all times (picture).

Let $w$ be the magnitude of the applicable acceleration this frame. It can be defined as a piecewise function as
\[
w = \begin{cases}
    \gamma_1 & \gamma_2 \ge \gamma_1\\
    \gamma_2 & \gamma_1 > \gamma_2 > 0\\
    0 & 0 \ge \gamma_2
\end{cases}
\]
where $\gamma_1$ represents the upper-bound of acceleration and is proportional to the engine constant \verb|sv_airacclerate| ($A$), and \verb|sv_maxspeed| ($L$); $\gamma_2$ represents signed the difference between $L$ and the projected velocity.

When $\gamma_2$ is negative, meaning the projected velocity is larger than the max speed, no acceleration will occur and $w=0$; when $\gamma_2$ is positive but below $\gamma_1$, meaning that adding the full acceleration will overshoot the max speed, only the difference between the max speed and the current projected velocity is applied, bringing the player to the speed limit on the next frame and $w=\gamma_2$; when $\gamma_2$ is larger than $\gamma_1$, only the upper-bound acceleration is applied and $w=\gamma_1$.

The definition of $\gamma_1$ and $\gamma_2$ are:
\begin{align*}
    \gamma_1 &= LA\tau\\
    \gamma_2 &= L - \tv \cdot \tunit{\ta},
\end{align*}
where the dot product $\tv \cdot \tunit{\ta}$ with the unit acceleration vector represents the projection of velocity.

\paragraph{Step 4} The new velocity on the next frame $\tv'$ (note that we assume step 1 has already been applied this frame) is the combination between the friction and movement acceleration:
\begin{align}
    \tv' &= \lambda(\tv) + w \tunit{\ta}\label{eq:dis_vel},
\end{align}
with the position being updated without interpolation ($\tp'$):
\[
\tp' = \tp + \tv' \tau.
\]

% jumping mechanics
% took 45 frames, v0 = 284hu/s, 60fps
Additionally, the mechanics of initiating a jump action can also be modelled mathematically. By recording myself jumping ingame at 60 frames per second (picture), I've obtained that a jump is represented by an impulse applied on the player, setting its z-axis velocity $v_{0z}$ to around $+280\pm10$ the frame after jumping. The player then takes around $45 + \pm1$ video frames to land, meaning an airtime of $45\pm1 / 60 =0.75\pm 0.02\si{s}$. Therefore the initial velocity is restricted by $\tv_z = 280\pm10$.

% summarize
To summarize, every $\tau$ seconds after $t=0$ the velocity is updated by gravity in the z-axis, and by the player's movement acceleration in the x,y-axis. The latter is limited so that the projection of the new velocity never exceeds the engine constant \verb|sv_maxspeed| of 30 in air. Therefore for some time $t$, the velocity next frame $\tv(t+\tau)$ is
\begin{align*}
    \tv(t+\tau) &= \tang{\lambda(\tv_x(t)) + w \tunit{a}_x, \lambda(\tv_y(t)) + w\tunit{a}_y,\tv_z(t) + g\tau}\\
    \tp(t+\tau) &= \tang{\tp_x(t) + \tv(t+\tau)_x \tau, \tp_y(t) + \tv(t+\tau)_y \tau, \tp_z(t) + \tv_z(t)\tau + \frac{1}{2}g\tau^2}.
\end{align*}

The velocity update can be represented in code for simulation, shown in appendix (citation).


\section{Basic Models}
I want to set some baseline movement models in this section, for it not only helps me to visualize the problem, but can also test my conjuncture in the negligibility of the continuous approximations.

My approach to maximize player jumping displacement can be described in sections. Firstly, I wish to find an optimal starting condition, then an analogy can made with projectile motion, leading to the precalculation of the vertical motion; Secondly, I hope to model the mechanics between the remaining two axis of motion; Lastly, I will fit various acceleration models base on my intuition, calculus, and even the switching to polar coordinates.

% straight line
\subsection{Key idea and boundary condition}
The key idea that will be repeated throughout this analysis is the extend that maximizing the player velocity will maximize the displacement. I hypothesized this from the position differential equations in Eq. \ref{eq:1de2}, that under the assumption of constant velocity, the final displacement will be proportional to that constant:
\begin{align*}
    \tp'(t) &= \tv(t)\\
    \tp(t) &= \int \tv(t) \, dt\\
    &= t(\tv(t)) + \tp(0) = t(\tv(t)) + \tvec{0},\\
    \tp &\propto \tv(t).
\end{align*}

Therefore, I was able to set an optimal initial velocity for the player, also known as the boundary condition of the problem. For we are maximizing the jump displacement, and higher velocity equates to higher displacement, it is reasonable to set the initial velocity as high as possible. So
\[
    \tmag{\tv} = 250,
\]
where $250$ is fastest ground running speed.


\begin{figure}[H]
    \centering
    \begin{minipage}{.5\textwidth}
        \centering
        \includegraphics[width=0.9\linewidth]{assets/1coordinates.png}
        \caption{The coordinate system}
        \label{fig:1coordinates}
    \end{minipage}%
    \begin{minipage}{.5\textwidth}
        \centering
        \includegraphics[width=0.9\linewidth]{assets/2coordinate_topdown.png}
        \caption{The top down view}
        \label{fig:2coordinates_topdown}
    \end{minipage}
\end{figure}


Furthermore, it should be noted that the problem is rotationally invariant when viewed top down. Let the z-axis denote the vertical height of the player, and the x,y-axis to be the plane displacement from a top down view (see figure \ref{fig:1coordinates}, figure \ref{fig:2coordinates_topdown}). \begin{wrapfigure}{r}{0.40\textwidth}
    \includegraphics[width=0.37\textwidth,right]{assets/2turning.png}
    \caption{Rotational Invariant (the player wishes to go rightwards)}
    \label{fig:2turning}
\end{wrapfigure}
By considering only the x,y-axis --- and the z-axis remaining constant before and after the jump --- I realized that an optimal strategy does not depend on the angle of the initial velocity vector within the plane, for unless we somehow ended up behind our initial direction, it is always possible to rotate the initial velocity such that we travel in a desired direction of displacement (see figure \ref{fig:2turning}). This meant that the optimal initial velocity is not restricted to a certain direction, but for the sake of consistency between models, I shall use the initial velocity pointing in the y-axis:
\[
    \tv = \tang{0, 250, 0}.
\]

\subsection{Straight Line}

With the key idea and boundary condition in mind, I attempted a straight line strategy. This model has the player accelerate and move in a straight line, for mathematics has taught me that the shortest distance between two points comes from a line, resulting in less time needed to travel a long distance, which is helpful in optimizing displacement over time. To achieve this, the player would need to accelerate along the y-axis, for that is in the direction of its initial velocity, which maximizes speed at all times, and would theoretically result in the largest displacement. Mathematically, this is equivalent, to setting the directional vector to
\[
    \td(t) = \tang{0, 1, 0}
\]
for all time $t$. Realistically, this will be equivalent to pressing down ``w'' and not moving the mouse.
% do model!!
\begin{figure}[H]
    \centering
    \begin{minipage}{.5\textwidth}
        \centering
        \includegraphics[width=0.9\linewidth]{assets/2straightjumping.png}
        \caption{Straight Jumping}
        \label{fig:2straightjumping}
    \end{minipage}%
    \begin{minipage}{.5\textwidth}
        \centering
        \includegraphics[width=0.9\linewidth]{assets/straight_constraint.png}
        \caption{Simulated Player Movement}
        \label{fig:straight_constraint}
    \end{minipage}
\end{figure}

To test this model, I recorded myself jumping and doing the actions described above --- it is important that the x,z-axis should not be changed after the jump; it is only the y-axis that I am recording. Figure \ref{fig:straight_constraint} shows my estimation of the real player's path with software; Figure \ref{fig:2straightjumping} shows the coordinates of the player at the start of the jump $\tp(0)$, the first discrete time of the jump $\tp(\tau)$, and the end of the jump $\tp(t_f)$ --- the symbol $t_f$ denotes the duration of the jump. Here are them written inline:
\begin{align}
 \tp(0) &= \tang{1702.63, 725.59, 128.03}, \quad \tmag{\tv(0)} = 250.00 \label{eq:2emp0}\\
 \tp(\tau) &= \tang{1702.63, 721.27, 130.83}, \quad \tmag{\tv(\tau)} = 378.35 \label{eq:2emp1}\\
 \tp(t_f) &= \tang{1702.65,541.95, 132.94}, \quad \tmag{\tv(t_f)} = 250.00 \label{eq:2emp2}.
\end{align}

\subsubsection{Analysis}
Using the simple empirical data in Eq. \ref{eq:2emp0}, Eq. \ref{eq:2emp1}, and Eq. \ref{eq:2emp2}, not only can I analyze the straight line model, but I can also calculate the various fundamental constants within the problem --- such as impulse and acceleration. However, I've also noticed that the x,z-axis has changed after the jump. The x change is likely the result of my uncertainty in having an initial velocity directly in the y-axis, but due to the rotational invariant discussed before, this is a non detrimental problem; but the z-axis is different and is likely to be the result of some ``bounciness'' of the player as it hits the floor --- which doesn't matter in the optimizing of displacement. Thus, I chose to ignore only the z-axis change during calculations.

The total displacement of this jump par definition is the expression $d = \tmag{\tp(t_f) - \tp(0)}$, which is
\begin{align*}
    d &= \sqrt{(\tp_x(t_f) - \tp_x(0))^2 + (\tp_y(t_f) - \tp_y(0))^2 + (\tp_z(t_f) - \tp_z(0))^2}\\
    &= \sqrt{(1702.65-1702.63)^2 + (541.95-725.59)^2 + (0)^2}\\
    &\approx 183.64.
\end{align*}


Firstly notice the sudden change in the velocity of the player at in first frame, time $\tau$. From this I deduced that the act of jumping is achieved by an impulse --- a sudden change in velocity --- upon the z-axis. With the assumption that no such impulses occurs in the x,y-axis, we can find the vertical change in velocity of the player at the start of the jump using the Pythagorean Theorem (figure \ref{fig:2verticalimpulse}):
\begin{wrapfigure}{r}{0.40\textwidth}
    \includegraphics[width=0.37\textwidth,right]{assets/2verticalimpulse.png}
    \caption{Jumping impulses}
    \label{fig:2verticalimpulse}
\end{wrapfigure}
\begin{align*}
    \tmag{\tv}^2 &= \tv_y^2 + \tv_z^2\\
    \tv_z &= \sqrt{\tmag{\tv}^2 - \tv_y^2}\\
    &= \sqrt{378.35^2 -250^2}\\
    &= 284.00.
\end{align*}

With approximations in continuous time, we can set the player's initial vertical velocity to this number such that
\[
    \tv_{z}(0) = 284.
\]


Secondly, by measuring the duration of the jump --- which I've done by counting the number of video frames of a sixty hertz video, or $44 \times \frac{1}{60} = 0.7333 \si{s}$ --- I was able to derive the average speed $\tv_{a}$ of the player:
\[
    \tv_{a} = \frac{183.71}{0.7333} \approx 250.42.
\]

Isn't that surprising? Turns out that there is almost zero net acceleration in the straight line model --- the average velocity does not differ from its initial value --- which meant that the acceleration as a function of time $t$ and wishing direction $\td(t)$ is zero:
\[
    \ta(t, \td) = \tang{0, 0, 0}.
\]

I suspect a speed limit is in play here. This is bad in the optimization of jump displacement as it restricts the player's velocity, which by the key idea, is likely to limit the displacement too. But the game certainty cannot be limiting the magnitude of velocity directly as I've achieved higher velocity while playing, often subconsciously. My research in the engine source code \parencite{valvesoftware} and a simplified article about the mechanics in a similar game called ``Half Life'' \parencite{jwchong} was targeting exactly this mechanic of velocity limiting. In short, the player's velocity $\tv$ is limited by its projection onto the wished direction $\td$.

\subsubsection{Speed limit}
\begin{wrapfigure}{r}{0.40\textwidth}
    \includegraphics[width=0.37\textwidth,right]{assets/2proj.png}
    \caption{Projection limiting}
    \label{fig:2proj}
\end{wrapfigure}
A vector's projection onto another is another way to say how much the vector is pointing in the direction of another, in the form of a number. Denoted by $\text{proj}(\tv, \td)$, the projection is the side of the triangle along the direction of $\td$ form by a perpendicular line from $\tv$ to $\td$ (figure \ref{fig:2proj}). This can be defined using dot products by the following identity (APPENDIX proof):
\[
    \text{proj}(\tv, \td) = \frac{\tv \cdot \td}{\tmag{\td}}.
\]

Additionally, the game engine states that the vector $\td$ never has a z-component and is a unit vector, meaning that its magnitude is zero. Therefore the projection is simply the 2D vector dot product between $\tv$ and $\td$.
\begin{align*}
    \text{proj}(\tv, \td) &= \tv \cdot \td\\
    &= \tv_x \td_x + \tv_y \td_y + \tv_z \td_z\\
    &= \tv_x \td_x + \tv_y \td_y.
\end{align*}

Lastly, the game engine computes the player's 2D (x,y-axis) acceleration by checking if this projection excesses a limit $L$, set to $250$. If it does, acceleration will be zero; else, acceleration is a constant in the direction of $\td$, equal to $LA\tau$ in the discrete sense, and $LA$ in the continuous approximation --- where $A$ is a multiplier equal to $10$. The player's 2D acceleration function can therefore be defined piecewisely as:
\begin{figure}[H]
    \centering
    \[
        \ta(t) = \begin{cases}
            LA \td & \text{proj}(\tv, \td) < L\\
            0 & \text{otherwise}
        \end{cases}
    \]
        \caption{Player 2D acceleration function}
    \label{eq:playeracceleration}

\end{figure}

In the case of maximizing displacement, we ideally want the first condition of equation \ref{eq:playeracceleration} to be true for as long as possible. Yet in the context of my straight line model, it is simple to see that the first condition is never true, for
\begin{align*}
    \text{proj}(\tv, \td) &= \tv_x \td_x + \tv_y \td_y\\
    &= \tv_x \times 0 + \tv_y \times 1\\
    &= \tv_y,\\
    \text{and because} \quad \tv_{y}(0) &= 250 = L\\
    \text{proj}(\tv, \td) &= L\\
    \text{and} \quad \ta(t) &= 0.
\end{align*}
Our acceleration model has therefore confirmed our hypothesis that the player is not accelerating in the x,y-axis at all. This makes the model a terrible one for maximizing one's jumping displacement, as low acceleration equals to lower velocity, and in turn decreases displacement per my key idea.

\subsubsection{Continuous modeling approximation}
But before I try some other models, let's first evaluate an continuous approximation using this simplistic model.
The game engine's code reveals a gravitational acceleration $g$ --- with a value of $800$ --- on the player at all times, and I used this as an opportunity to test my continuous approximation by comparing the calculated value with the real value. Furthermore, because the z-axis acceleration is not dependent on the $\td$ --- the player's actions, we shall only attempt to model the x,y-axis actions of the player.

The objective is to solve the differential equations of motion for the player. For the x,y-axis acceleration of the player is zero, and let $g$ be the signed z-axis acceleration:
\[
    \ta = \tang{0, 0, g},
\]
the velocity as defined in Eq. \ref{eq:1de1} is the integral of acceleration --- note that $\tv_{z}(0) = 284$ as calculated before:
\begin{align*}
    \tv &= \int \ta \, dt\\
    &= \tang{0, 0, gt} + \tv(0)\\
    &= \tang{0, 250, gt + 284}.
\end{align*}

Continuing, the player position is the integral of velocity as shown in Eq. \ref{eq:1de2}. By assuming that $\tp(0) = 0$,
\begin{align*}
    \tp &= \int \tv \, dt\\
    &= \tang{0, 250t, \frac{1}{2}gt^2 + 284t} + \tp(0)\\
    &= \tang{0, 250t, \frac{1}{2}gt^2 + 284t}
\end{align*}
shows that the the y-axis displacement is $250t$. Because I've calculated the jump displacement, this means that the jump duration $t_f$ is
\[
    t_f = \frac{183.64}{250} \approx 0.735 \si{s}.
\]
Thus, since the z-axis displacement is zero, the gravitational acceleration $g$ is calculated to be
\begin{align*}
    \tp_{z}(t_f) &= \frac{1}{2}gt_f^2 + 284t_f = 0\\
    g &= - 2 \times \frac{284t_f}{t_f^2}\\
    &= -\frac{568}{t_f}\\
    &\approx -772.79.
\end{align*}

Comparing to the actual value of $g=800$, the absolute value of $772.79$ is not far off. The relative error of $\frac{800 - 772.79}{800} = 3.4\%$ is close enough for an approximation. Therefore it is justified that for the rest of the models that I can continuously use the continuous approximations, to which I will.

\subsection{Skilled Player's model}
A better strategy would be one where the player accelerates not directly along their velocity, but rather at an angle to their velocity. Personally I've been told by numerous skilled players that you need to move your mouse slowly to the right --- meaning rotating the vector $\td$ over time --- to get higher speed and distance. Now I can test this theory.

Currently, any strategy is uniquely defined by the player wishing direction function $\td(t)$. Therefore let
\[
    \td(t) = \tang{x(t), y(t)}
\]
where $x(t)$ and $y(t)$ are the x,y-components of the function. Ideally, this vector should be y-axis pointing at $t=0$, and rotate clockwise overtime as to create acceleration rotation. This reminded me of a unit circle, albeit with the x,y-axis switched, creating
\[
    \td(t) = \tang{\sin(t), \cos(t)}.
\]
One can verify that this definition both denote rotation overtime and have a magnitude of $1$. Furthermore, I can encode the rate of rotation by scaling the inner $t$ in the two trig functions by the same factor $w$, without changing its magnitude. Therefore the player direction function $\td(t)$ as recommended by skilled players is
\[
    \td(t) = \tang{\sin(wt), \cos(wt)}.
\]

My task is to find the best constant $w$ that maximizes this function. The plan is to solve the x,y-axis differential equation into a function of position $\tp$, then explicitly writing the speed limit as an inequality constraint, and finally selecting the constant $w$ that (hopefully) satisfies the constraint and maximizes the displacement.

\subsubsection{Displacement equation}
The key idea suggests that I should maximize the velocity, and as explained in the previous subsection, this meant that player acceleration $\ta$ is to be maximized by fulfilling the inequality projection in equation \ref{eq:playeracceleration}. Thus by assumption that this inequality is held,
\[
    \ta(t) = LA\td.
\]

Let $k=LA$. Notice that the velocity is the integral of acceleration; we also have to satisfy the initial velocity at $t=0$. So,
\begin{align*}
    \tv(t) &= \int \ta \, dt\\
    &= \int k \tpar{\sin(wt)}{\cos(wt)} \, dt\\
    &= \frac{k}{w} \tpar{-\cos(wt)}{\sin(wt)} + c,\\
    \tv(0) &= \frac{k}{w} \tpar{-\cos(0)}{\sin(0)} + c\\
    c &= \frac{k}{w} \tpar{1}{0} + \tpar{\tv_x(0)}{\tv_y(0)}.
\end{align*}

And the position is the integral of velocity. Again, assuming that the initial displacement is $0$. So,
\begin{align*}
    \tp(t) &= \int \tv \, dt\\
    &= \int \frac{k}{w} \tpar{-\cos(wt)}{\sin(wt)} + \frac{k}{w}\tpar{1}{0} + \tpar{\tv_x(0)}{\tv_y(0)} \, dt\\
    &= \frac{-k}{w^2} \tpar{\sin(wt)}{\cos(wt)} + \frac{k}{w}\tpar{t}{0}  + t\tpar{\tv_x(0)}{\tv_y(0)} + c.
\end{align*}

To find the integration constant $c$, consider the x-axis displacement function
\begin{align*}
    \tp_x(t) &= \frac{-k}{w^2} \sin(wt) + t\tv_x(0) + t\tv_x(0) + c_x = 0\\
    c_x &= 0,
\end{align*}
and the y-axis displacement function
\begin{align*}
    \tp_y(t) &= \frac{-k}{w^2} \cos(wt) + t\tv_y(0) + c_y = 0\\
    c_y &= \frac{k}{w^2}.
\end{align*}

Using the Pythagorean Theorem, the magnitude of the player's displacement using this model after a jump --- where $t=t_f$ --- is:
\begin{figure}[H]
    \centering
    \begin{align*}
        \tmag{\tp(t_f) - \tp(0)} &= \sqrt{\tp_x(t_f)^2 + \tp_y(t_f)^2}\\
        &= \sqrt{\left(\frac{-k}{w^2}\sin(wt_f) + t_f\frac{k}{w} \right)^2 + \left( \frac{-k}{w^2}\cos(wt_f) + 250t_f + \frac{k}{w^2} \right)^2}
    \end{align*}
    \caption{Skilled Displacement}
    \label{eq:2skilled_displacement}

\end{figure}

\begin{figure}[H]
    \centering
    \includegraphics[width=0.85\textwidth]{assets/skilled_displacement.png}
    \caption{}
    \label{fig:skilled_displacement}

\end{figure}
The graph of equation \ref{eq:2skilled_displacement}, shown in figure \ref{fig:skilled_displacement}, is symmetrical along the y-axis. This is because of yet another symmetry between a left-turning and a right-turning jump. Quantitatively, this ``unlimited'' strategy peeks when $w=0$, where the strategy collapses to my straight line model. As the turning speed $w$ increases, the ideal displacement decreases and levels near $w=10$: I guess that a spinning player does not contribute much to jumping far. Therefore we need to choose the smallest $w$ possible to maximize potential jump displacement.

\subsubsection{Restriction equation}
In addition to finding the ``unlimited'' displacement of the strategy for all constants $w$, we also need check if they satisfy the assumption of unrestricted acceleration at all times. Ideally there should be a value of $w$ that always fits this constraint, but the smallest ``error'' from the ideal is always a backup option.

As shown in figure \ref{eq:playeracceleration}, for maximum acceleration of $\ta(t) = LA\td$, the constraint
\[
    \text{proj}(\tv, \td) < L
\]
must be held for the duration of the jump from $t=0$ to $t=t_f$. Expanding and organizing this inequality yields
\begin{align*}
    L &> \text{proj}(\tv, \td)\\
    &> \tv_x \td_x + \tv_y \td_y\\
    &> \left(-\frac{k}{w} \cos(wt) + \frac{k}{w}\right) \sin(wt) + \left(\frac{k}{w} \sin(wt) + 250\right) \cos(wt),\\
    0 &> \left(-\frac{k}{w} \cos(wt) + \frac{k}{w}\right) \sin(wt) + \left(\frac{k}{w} \sin(wt) + 250\right) \cos(wt) - 250
\end{align*}

To find the value of the constant $w$ that best fulfills this inequality, I need a quantitative measure on how well this inequality is held through the jump. Ideally the function should only include the failures of fulfilling the inequality, but as explained in the more advanced models --- that the projection RHS should be as close to zero as possible for high speeds --- I chose an integral from $t=0$ to $t=t_f$ on the absolute values of the RHS, denoted as the ``error'' of a model. This is done with the absolute function $|x|$ defined by
\[
 |x| = \begin{cases}
         x & x \geq 0\\
         -x & x < 0
        \end{cases}
\]

as to make all errors positive. Therefore the error function $R(w)$ for each constant $w$ is the integral of the absolute value of the RHS over the jump:
\begin{figure}[H]
 \centering
 \[
  R(w) = \int_0^{t_f} \left|\left(-\frac{k}{w} \cos(wt) + \frac{k}{w}\right) \sin(wt) + \left(\frac{k}{w} \sin(wt) + 250\right) \cos(wt) - 250\right| \, dt
 \]
 \caption{The error function}
 \label{eq:2error}
\end{figure}


\begin{figure}[H]
 \centering
 \includegraphics[width=0.85\textwidth]{assets/restriction_equation.png}
 \caption{}
 \label{fig:2error}
\end{figure}
% TODO: if needed, show the two red dots and choose the smallest one
Figure \ref{fig:2error} shows the graph of the error function in equation \ref{eq:2error}. While it may be possible to find the minimum of the error function using calculus, the steps are too complex and a numerical is taken instead and is shown by the red dot in the graph.

\begin{wrapfigure}{r}{0.50\textwidth}
    \includegraphics[width=0.47\textwidth,right]{assets/skilled_player.png}
    \caption{Skilled Player's Model}
    \label{fig:skilled_player}
\end{wrapfigure}

My program shows that the red dot is at coordinate
\[
    \tang{4.2884, 184.4228},
\]
meaning that the error function is the smallest in this skilled strategy with a constant $w=4.2884$. So the skilled player's model is defined by the direction function
\[
    \td(t) = \tang{\sin(4.2884t), \cos(4.2884t)},
\]
indicating a relatively fast rate of turning at a period of $\frac{2\pi}{4.2884} = 1.4652\si{s}$.


Ideally, this optimized model corresponds to a displacement of $627.35$ using figure \ref{fig:skilled_displacement}. But I suspect the leftover ``$184.4228$'' errors to decrease the displacement by quite an amount. Figure \ref{fig:skilled_player} shows the actual path of the player when simulated in game; this is in combination with a correctly estimated lower displacement of $254.19$ units.
\begin{align*}
    \tp(t_f) &= \tang{88.3021, 238.3556}\\
    \tmag{\tp(t_f)} &\approx 254.19
\end{align*}

The skilled player's model does indeed produce a higher displacement ($254.19$) than the straight line model ($183.64$). Not only does this confirm the theory from these skilled people, but the large error of the even optimized model could indicate further possible optimizations when I look at the problem discretely.

\section{Models}
With the question defined, the equations of motion derived, we can now consider the different models to achieve maximum displacement from a single jump.

% TODO: maybe put the constants before engine section
\subsection{Constants}
One might notice that I've deliberately left some of the constants $j_i$ clear of meaningful values and only suggested in-game default values. This is because many long-jumpers in the community find the default (vanilla) settings of the game engines ``bland'', reducing the excitement of this activity. Therefore, I will be setting some of the engine constants myself.

There are 4 of them:
\begin{itemize}
    \item \verb|sv_gravity| ($g$), set to 800
    \item \verb|sv_maxspeed| ($L$), set to 250
    \item \verb|sv_airaccelerate| ($A$), set to 12
    \item \verb|tickrate| ($n$), set to 64, or $\tau=\frac{1}{64}$
\end{itemize}

The only meaning deviation from the default settings are the speed limit $L$, for the default settings limits the optimization one can achieve, while the higher value of $250$ simply magnifies the result that one can achieve.

Furthermore, all models are free to set the initial velocity $\tv_0$, with the constraint that the magnitude of $\tv_0$ does not exceed $250$, the speed limit.

\subsection{Straight line}
% where you do nothing
We all know the fastest route (shortest distance) from point $A$ to point $B$ is with a straight line connecting $A$ to $B$. Therefore if the velocity magnitude is maximized, and as time is proportional to distance: $t = \frac{s}{\tmag{\tv}}$, it seems that a straight line would maximize the jumping distance assuming no constraints. This will serve as a good baseline for all the other models.

I have decided for the player to travel along the y-axis because I personally find it nicer to graph and draw with, but all maths would apply if you wish to substitute for the x-axis. This is represented by the acceleration function:

\begin{figure}[H]
    \centering
    \[
    \tunit{a}(t) = \tang{0, 1}
    \]
    \caption{The straight line model}
\end{figure}

Notice that player acceleration is constant in this model, and which can be achieved simply by looking forward, and press the key $w$ during the jump. Furthermore, we can set the initial velocity to point directly upwards in the y direction, such that:
\[
\tv_0 = \tang{0, 250},
\]
for common knowledge tells me that a faster initial speed in the direction where I am going will result in higher overall travel distance. Overall, there exists four metrics for each model, each worth evaluating.

We can first look at the case without the max speed constraint. By running a simulation with the engine constants discussed before, I got a result of $\approx 933.84$ hammer units in a jump. Predictably, this model will have the player traveling in a straight line (figure \ref{fig:straight_nothing_1}).

\begin{wrapfigure}{r}{0.48\textwidth}
    \includegraphics[width=0.45\textwidth,right]{assets/straight_nothing_1.png}
    \caption{}
    \label{fig:straight_nothing_1}
\end{wrapfigure}

I can comfortably say that this jumping distance would never be possible to achieve in game, for this model can be proven to be the most optimal jumping path if there are no speed limits (appendix, using EL equation). This number of around $900$ hammer units serves to be the upper-bound output of our optimization, and the lower-bound being its negative of around $-900$ (if you wish to accelerate directly backwards with negative initial velocity).

Additionally we can also utilize the equation derived in the last section (equations \ref{eq:jumping}).

Using the acceleration function, we can define $x(t)$ and $y(t)$ as:
\[
    x(t) = 0, \quad y(t) = 1.
\]

Their first and second order indefinite integrals can be obtained using calculus:
\begin{alignat*}{3}
    X(t) &= \int x(t) \, dx = 0, \quad
    &&\tfx(t) &&= \int X(t) \, dx = 0\\
    Y(t) &= \int y(t) \, dx = t, \quad
    &&\tfy(t) &&= \int Y(t) \, dx = \frac{1}{2} t^2,
\end{alignat*}
notice that the constants during integration are omitted as we've incorporated them into the derivation of the jumping motion equations as $c_1$, and $c_2$.

Therefore the $x$ position after $t_f$ seconds assuming no speed limit is:
\begin{align*}
    p_x &= p(t_f)_x = w\tfx(t_f) + t_f(v_{0x} - w\tfx'(0)) - w\tfx(0)\\
    &= w \times 0 + t_f(0 - w \times 0) - w \times 0\\
    &= 0,
\end{align*}
and the respective $y$ position is:
\begin{align*}
    p_y &= p(t_f)_y = w\tfy(t_f) + t_f(v_{0y} - w\tfy'(0)) - w\tfy(0)\\
    &= w \frac{1}{2} t_f^2 + t_f(v_{0y} - w v_{0y} \times 0) - w \times 0\\
    &= (12 \times 250) \frac{1}{2} (0.7)^2 + 0.7 \times 250\\
    &= 910.
\end{align*}

% TODO: change A to 10

The jumping motion differential equations resulted in a similar sized jumping distance of $910$ is comparison to the engine simulated $\approx934$. I believe that a $\frac{|910-934|}{934} \approx 2.57\%$ error is a very well approximation of this continuous method for future modeling.

\begin{wrapfigure}{r}{0.48\textwidth}
    \includegraphics[width=0.45\textwidth,right]{assets/straight_constraint_inequality.png}
    \caption{}
    \label{fig:straight_constraint_inequality}
\end{wrapfigure}

But the game analyzed does have the constraint of a speed limit so players cannot accelerate infinitely, traveling across the map in two-to-three seconds. The graphing of the constraint (equation \ref{eq:constraint}) shows the problems (figure \ref{fig:straight_constraint_inequality}).



Blah blah blah one line is higher then the other,

% TODO: calculated the restricted line, maximum 250 words,

% TODO: conclude this: max 100 words

% TODO Have around 1500 words

% Do the machine learning one

% Do the step by step one

% Do the euler-lagrange one

% Extension, find the optimal angle for each




\section{Conclusion}
There were four distinct models analyzed in this essay: the straight line model, the skilled player's model, the first difference equation, and the second difference equation\footnote{Graph making and numerical calculations code: \url{https://github.com/Troppydash/SchoolCDN/blob/master/MATHS_EE.ipynb}}. Each model is optimized towards a different numerical attribute of the system: straight line model optimizes ``unrestricted'' displacement, skilled player's model optimizes turning, first difference equation optimizes velocity, and second difference equation optimizes total displacement. Coincidentally they are ranked in quality and complexity from low to high (table \ref{tbl:dis}), with the easiest one requiring the player to do nothing except a running start, and the most complex one needing constant monitoring of the velocity vector.

\begin{figure}[H]
    \centering
    \begin{tabular}{|c|c|c|c|c|}
        \hline
        & Straight Line & Skilled Player's & 1st Diff. Eq. & 2nd Diff. Eq.\\
        \hline
        Displacement & 183.64  & 254.19 & 373.00 & 392.15 \\
        \hline
    \end{tabular}
    \caption{Displacements from the models}
    \label{tbl:dis}
\end{figure}

Method-wise, the continuous approximation models were appealing at the start as I intended to use calculus to optimize the problem; but as the velocity limiting piecewise function creeped in, I found the discrete methods to be less complicated and more rewarding. Additionally, I ended up disproving my key idea in that higher velocity may not equate to higher displacement, and subsequently for higher acceleration as well. Breaking the key idea might allow further optimizations of the Skilled player's model in choosing a better $w$, and the discrete models in allowing some deceleration, all of which could improve jumping displacement further. The true degree to which this may undermine this analysis is unknown, but if I were to come back to this problem, I have some ideas in functional maximizing the displacement function directly using variational calculus \citefoot{apushkinskaya_2014}.

To summarize, this essay explored the optimal jumping displacement of a player in ``Counter Strike, Source''. I first modeled the game with empirical data from the straight line model, deriving the mechanics behind jumping and researching the speed limit. In a case to avoid the speed limit, I've tested theories from both the skilled people and myself, and created a modification to push the optimization further on the second discrete model. I believe that this exploration is useful towards the tool-assisted speedrun community, owing to the high amount of skill and precision needed for the advance models.




%[-2.2815926535898114, 99.94724809670372]
% do the modeling





% evantually my best, and as a model

% try it



%\section{Vertical motion}
Universal across the various metrics are the vertical motion of the player. Firstly notice that the z-axis motion is separated from the x,y-axis motion, meaning that it can be computed on its own.

Let $p$, $v$, $a$ be the z-axis position, velocity, and acceleration as a function of time $t$ and engine constants $S$, also define $p_0$, $v_0$ to be the initial conditions. Because of the velocity update in equation \ref{eq:gv}:
\[
\frac{\Delta v}{\tau} = g,
\]
because $\tau$ is very small, it can be assumed that:
\[
v' = g.
\]
Similarly, the position update can also be assumed to be continuous (equation \ref{eq:gp}):
\begin{align*}
    \frac{\Delta p}{\tau} &= v(t+\tau)\\
    p' &= v.
\end{align*}
Therefore it is possible to compute the airtime of the player upon a jump action. For the acceleration of gravity is constant, the kinematic equations may be used (citation).
\begin{align*}
    s &= s_0 + ut + \frac{1}{2} a t^2\\
    0 &= 0 + v_0 t + \frac{1}{2} g t^2\\
    \text{therefore:}&\\
    t &= \frac{-v_0 \pm \sqrt{v_0^2}}{g}\\
    &= \frac{-2v_0}{g} \,\, \text{or} \,\, 0.
\end{align*}
Because $t=0$ is the first frame when the player executes the jump action, the airtime is the first root of the quadratic, and therefore:
\begin{align*}
    t &= -2 \frac{v_0}{g}\\
    &= -2 \times \frac{280\pm 10}{-800}\\
    &= 0.70 \pm 0.03 \si{s}
\end{align*}

The result is very similar to the recorded airtime of $0.75\pm 0.02$ with their uncertainty overlapping, the assumption of a small $\tau$ may also contribute to the difference between the observed with the calculated time. This airtime will be denoted by the symbol $t_f$.

Therefore if the initial displacement at $t=0$ ($\tp(0)$) has a length of zero, the total displacement as a result of the jump is $\tmag{\tp(t_f)}$.

%\newpage
%\section{Jumping motion}
% main is with straight jumping
I will consider and evaluate multiple models to maximize the displacement jumped, this will include the modeling of the player motions as an differential equation assuming continuous time, then going discrete and maximize the displacement frame by frame.

\subsection{Equations of motion}
First let us construct the differential equations that govern the player motion, maybe the answer is as simple as an optimization problem. Let $\ta$, $\tv$, and $\tp$ be the two dimensional acceleration, velocity, and position. It is assumed that player can perfectly control $\ta$ at all time $t \in [0, t_f]$ via their keyboard and mouse --- the time period $t\in[0,t_f]$ will be assumed for all functions in this section.

Initially consider the case without any maximum velocity restrictions, the acceleration is:
\begin{align*}
    \ta = w\tunit{a} = w\tpar{x(t)}{y(t)},
\end{align*}
where $x(t)$ and $y(t)$ are functions of time and is a unit vector where
\[
x^2 + y^2 = 1.
\]

The value of $w$ is different under continuous assumptions, albeit still reminding constant throughout. We will define $w$ by the maximized acceleration of the player divided by the change in time, that is: $w = \frac{\gamma_1}{\tau}$. The justification is that the discrete scalar $w$ can be considered as a tiny change in velocity, which can be rearranged (notice that this uses equation \ref{eq:dis_vel} where $w=\gamma_1$):
\begin{align*}
    d\tv &=  \tv' - \tv\\
    &= (\tv + \gamma_1 \tunit{a}) - \tv\\
    &= \gamma_1 \tunit{a}\\
    \text{therefore:}&\\
    \frac{d\tv}{\tau} &= \frac{\gamma_1}{\tau} \tunit{a},
\end{align*}
and as $\tau \to 0$, the equation will approach the time derivative of the velocity (acceleration) of the player, where the constant scalar $w$ is now defined by:
\[
w = \frac{\gamma_1}{\tau} = LA.
\]

Now notice that the velocity at any point is the time integral of the acceleration, which can be expanded into the x,y components like so:
\begin{align*}
    \tv &= \int w\tunit{a} \,dt + c\\
    &= \tpar{\int w x(t) \,dt}{ \int w y(t) \,dt} + \tpar{c_x}{c_y}\\
    &= w\tpar{X(t)}{Y(t)} + \tpar{c_x}{c_y},
\end{align*}
where $X(t)$ and $Y(t)$ are the time integrals of $x(t)$ and $y(t)$.

The constants ($c_0$,$c_1$) can be found by plugging in $t=0$, for we can define the constants in terms of the initial velocities:
\begin{align*}
    v_0 &= w \tpar{X(0)}{Y(0)} + \tpar{c_x}{c_y}\\
    c_x &= v_{0x} -wX(0)\\
    c_y &= v_{0y} -wY(0).
\end{align*}

The displacement $\tp$ is the time integral of the velocity:
\begin{align*}
    \tp &= \int \tv \, dt + k\\
    &= w \tpar{\int X(t) \, dt}{\int Y(t) \, dt} + \tpar{tc_x}{tc_y} + \tpar{k_x}{k_y}\\
    &= w \tpar {\tfx(t)}{\tfy(t)}  + t\tpar{c_x}{c_y}  + \tpar{k_x}{k_y},
\end{align*}
where $\tfx$ and $\tfy$ are the time integral of $X$ and $Y$.

The constants $k_x$ and $k_y$ can be found by evaluating at $t=0$:
\begin{align*}
    \tp_0 &= w \tpar{\tfx(0)}{\tfy(0)} + \tpar{k_x}{k_y} = \tpar{0}{0}\\
    k_x &= -w\tfx(0)\\
    k_y &= -w\tfy(0).
\end{align*}

Therefore, the position of the player, when $L >> 0$, at all time $t$ in the respective domain is show in figure \ref{eq:jumping} (notice that $\tfx'(t) = X(t)$ and $\tfy'(t) = Y(t)$):

\begin{figure}[H]
    \centering
    \begin{align*}
        \tp(t)_x &= w\tfx(t) + t(v_{0x} - w\tfx'(0)) - w\tfx(0)\\
        \tp(t)_y &= w\tfy(t) + t(v_{0y} - w\tfy'(0)) - w\tfy(0).
    \end{align*}
    \caption{The Jumping Motion equations}
    \label{eq:jumping}
\end{figure}


\subsection{Restrictions}
Step 3 states that the velocity is limited through the projection of the current velocity onto the player acceleration. I thought to define the restrictions so that maximum acceleration magnitudes are achieved at all time.

The maximum magnitudes of velocity after every frame is achieved when $L- v \cdot \tunit{a} \ge LA\tau$, for then $\gamma_2 \ge \gamma_1$ and $w = \gamma_1$, for this ensures that acceleration is at a maximum. Additionally, we can see that both the $x$ and $y$ position are positively correlated to $w$, meaning that its maximization will result in higher overall displacement. However, I wonder if this is accurate for the new velocity $\tv'$ is the sum of current velocity with the acceleration, the angle between $\tunit{a}$ and $\tv$ dictates the magnitude of the updated velocity --- could it sometimes be worth it to forgo maximum acceleration to decrease the angle, resulting in a higher updated velocity?

For now, consider the restriction so that acceleration will be maximized at all $t$ in the time domain:
\begin{align*}
    L - v \cdot \tunit{a} &\ge LA\tau\\
    L - (v_x \tunit{a}_x + v_y \tunit{a}_y) &\ge LA\tau.
\end{align*}

Notice that $\tunit{a}_x = x(t)$ and $\tunit{a}_y = y(t)$ and:
\begin{align*}
    v_x &= wX(t) + c_x\\
    v_y &= wY(t) + c_y,
\end{align*}
therefore:
\begin{align*}
    LA\tau\ &\le L - x(t)(wX(t)+c_x) - y(t)(wY(t)+c_y)\\
    L - LA\tau &\ge x(t)(wX(t)+c_x) + y(t)(wY(t)+c_y)\\
    L - LA\tau &\ge x(t)(wX(t) + v_{0x} -wX(0)) + y(t)(w(Y) + v_{0y} -wY(0))\\
    \frac{L - LA\tau}{w} &\ge X'(t)(X(t) + \frac{v_{0x}}{w} - X(0)) + Y'(t)(Y(t) + \frac{v_{0y}}{w} -Y(0)),
\end{align*}
because $w=\gamma_1=LA\tau$, and $x(t) = X'(t), \quad y(t) = Y'(t)$, the simplified constraint of the jumping motion equations is displayed in figure \ref{eq:constraint}.
\begin{figure}[H]
    \centering
    \[
        \frac{1}{A\tau} - 1 \ge X'(t)(X(t) + \frac{v_{0x}}{w} - X(0)) + Y'(t)(Y(t) + \frac{v_{0y}}{w} -Y(0))
    \]
    \caption{The Constraint Equation}
    \label{eq:constraint}
\end{figure}

This restriction shows that for acceleration each frame to be maximized, the RHS must be less than the LHS constant for all $t \in [0, t_f]$.

\subsection{Motion with restrictions}
To simulate the continuous motion described here, one can take advantages of the absolute function and its derivatives and integrals, along with computers to simulate definite and indefinite integrals.

% TODO: make this in program, or not, doesnt matter.
%\newpage

%\newpage


%\nocite{*}
\newpage
\addcontentsline{toc}{section}{References}
\printbibliography

%\newpage
%\appendix
%\counterwithin{figure}{section}
%\counterwithin{table}{section}
%\section{Image Creation Code}

%\input{appendix/code.tex}

\end{document}
