\documentclass[a4paper,12pt]{article}
\usepackage[utf8]{inputenc}
\usepackage{parskip}

\title{EE Research}
\author{Terry Qi}

\begin{document}

\maketitle
\newpage

\section{Research Question}

To what extent can a player travel in length through a jump action in vanilla conditions, within games made with the source engine?

\section{Planning}

\paragraph{Research Question} To what extent can a player travel in length through a jump action in vanilla conditions, within the game made with the source engine?

\paragraph{Signature} Alan Smith

\paragraph{Bibliography}
\begin{verbatim}
https://adrianb.io/2015/02/14/bunnyhop.html
https://en.wikipedia.org/wiki/Euler%E2%80%93Lagrange_equation
https://en.wikipedia.org/wiki/Lagrangian_mechanics
https://www.youtube.com/watch?v=VCHFCXgYdvY
\end{verbatim}

\paragraph{To Do List:}
Research the programming language and engine used for source.

Research the physics/programming behind the physics engine.

Create a simulation/game that simulates the physics.

Create my hypothesis and the scientific method, write my initial beliefs.

Identify the factors in the questions.

Simplify the question: Solve for one dimension with restrictions.

Solve for two dimensions with restrictions.

Solve for three dimensions --- final answer with restrictions.

Merge restrictions and find the solution

\paragraph{Key Questions}
How does a player travel in the source engine games?

How are length measured in the source engine?

What are the vanilla conditions or setting?

How to maximize or minimize distances in real life?

How to maximize or minimize distances in the engine?

\paragraph{Possible Sources}
The internet --- CSGO forums

Discord servers with source modders

Github source code of the open-sourced source engine

Youtube videos explaining the maths and physics of variational calculus

\paragraph{Arising Issues}
None so far

\section{Meeting Record}
Research and analyze the question, identify the required sources.

Focus on linear algebra, calculus and potentially some statistical models to approximate the solution.

Could consider adding some restrictions to keep the question linear and simple.

Student is confident in the direction of where to find their information.

\paragraph{First reflection}
In the first reflection session, I started by a revision of the previously considered topic, and proposed a different, more understandable topic. The topic, coming from a video game that I enjoy particularly and forms an integral part of my friendship group, I considered to be more worthwhile to investigate and applicable. It is the analysis of the distance one can jump in a particular genre of games, which is both an interesting statistic to derive and also a niche knowledge my friends and I can use during gaming sessions. For the meeting, I have outlined the list of subtopics to that I would need to explore and solve sequentially, in hopes of a pleasing analytic solution. My supervisor was accepting of my approach, but also urged me to consider a statistical numerical approach to the problem --- potentially through simulations. I believe that both methods are worthwhile to explore and will incorporate them into the research.

\end{document}
