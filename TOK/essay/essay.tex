\documentclass[a4paper,12pt]{article}
\usepackage[utf8]{inputenc}
\usepackage[a4paper,margin=1in,footskip=0.25in]{geometry}
\setlength{\parskip}{\baselineskip}

\usepackage{amsmath}
\usepackage{amssymb}

% fancy page numbers
\usepackage{fancyhdr}
% bottom right
\pagestyle{fancy}
\fancyhf{}
\fancyfoot[R]{\thepage}
\fancypagestyle{plain}{%
    \renewcommand{\headrulewidth}{0pt}%
    \fancyhf{}%
    \fancyfoot[R]{\thepage}%
}
%go
\renewcommand{\headrulewidth}{0pt}
\usepackage[style=apa]{biblatex}
\addbibresource{essay.bib}
\usepackage{hyperref}

% Footer citing

% this only cites author if author field is present
\DeclareCiteCommand{\citeauth}
{\boolfalse{citetracker}%
	\boolfalse{pagetracker}%
	\usebibmacro{prenote}}
{\iffieldundef{author}{}{(\printtext{\printfield{author},})}}
{\multicitedelim}
{\usebibmacro{postnote}}

\newcommand{\citefoot}[1]{\footnote{\citeall{#1}}}
\newcommand{\citeall}[1]{\citeauth{#1} \citetitle{#1} (\citeyear{#1}) publisher \citelist{#1}{publisher}}

\linespread{1.67}

\title{\vspace{-8ex}TOK Essay, Prompt 3}

\author{Word Count: 1600}
\date{}

\begin{document}

\maketitle
\begin{quote}
    Is it better to have questions that can't be answered than answers that can't be questioned? (Richard Feynman) Discuss with reference to mathematics and one other area of knowledge.
\end{quote}

The asking and answering of questions are undoubtedly unique tools of mankind as they probe deeper into the mechanics and implications of knowledge in the universe. Questions communicate uncertainties caused by new findings towards existing knowledge frameworks. Answers are arguments that alleviates the uncertainties posed by the question through logic or empirical evidences. Some answers can’t be questioned, for either they are logically derived from axioms, or are protected and enforced by society. Some questions can’t be answered, from its direct contradictions with existing knowledge or as it is subjective. It is important for knowledge seekers to understand the type of questions to ask and the type of answers to make, for more effective questions and answers will increase the rate of true knowledge creation that lead humans closer to the truth. In this essay, the exploration in the utilities between answers that can't be questioned against questions that can't be answered in Mathematics and Economics from the Human Sciences points to the perspective that it depends on the discipline the question or answer is raised in.

%With this essay, I will be exploring the utilities and preferences of answers that can’t be questioned against questions that can’t be answered in Mathematics and the discipline of economics from the Human Sciences.

%Due to the theoretical nature of Mathematics – where knowledge resides in the provable relationships between theorems and axioms – answers that can’t be questioned are better for the discipline as it enforces logical results that provides certainty in the area of knowledge.

The abstract nature of Mathematics that is purely theoretical makes answers that can't be questioned more beneficial as it enforces truthful logical results that increase the explanatory power of Mathematics, regardless of time. The focus on answers that must not be logically questioned separates less truthful hypotheses from correct and trustworthy theorems. On the other hand, the searching and attempting of questions that can’t be answered provides little impact, as it wastes time researching either subjective questions or paradoxes with no logical answers. Historically, this axiomatic method\citefoot{axiomatic} generated powerful truthful knowledge (like the Pythagorean Theorem) that transcends time, more specifically, the case of two similar hypotheses demonstrated the benefits of a convention to build knowledge through answers that can’t be questioned. Fermat's hypothesis suggested a lack of solutions for a specific polynomial equation, namely:
\[
a^n + b^n = c^n, \, \text{where} \,\, a, b, c \in \mathbb{Z} \quad n \ge 3
\]
Similarly, Euler’s conjecture --- a generalization of Fermat’s hypothesis --- also suggested a lack of solutions for a generalized polynomial equation, namely:
\[
a_1^k+a_2^k + ... + a_n^k = b^k, \, \text{where} \,\, a_i, k \in \mathbb{Z} \quad n \ge k
\]
Without proper unquestionable proofs, the statements, albeit seemly true for numerous cases, are not accepted as absolute truth for around 250 years\citefoot{eulers}. Interestingly, while Fermat's hypothesis was eventually proved by Andrew Wiles axiomatically, and the answer that is the proof creating new fields of study for its confirmation of other dependent conjectures\citefoot{wiles_1995}; the conjecture by Euler was disproved by a counterexample from a computer brute-force search\citefoot{lander_parkin_1966}. In essence, this dual example displayed how the truthfulness of even similar formed statements can differ. It demonstrated how without an axiomatic proof that can’t be questioned, human intuition may view Euler’s conjecture just as correct as Fermat's, therefore creating erroneous information in Mathematics. The acceptance of Euler’s conjecture and general mathematics without unquestionable proofs is problematic for it allows exploration based on fundamentally false information that can ultimately question theorems that are inherently correct. In contrast, the axiomatic mathematical knowledge seeker can trust that their contributions through the same axiomatic system have the same validity and are as unquestionable as existing results throughout time. The focus on proofs – answers that can’t be questioned – ultimately enforces rigor in mathematical knowledge that reduces wasted time and resources spent on unrewarding, questionable knowledge fields.

Yet, the exploring and asking of questions that can’t be answered can still provide crucial knowledge, particularly against the axioms of an axiomatic system. Although theorem and corollaries are answers that cannot be questioned logically, the selection of the subjective axioms definitely can if the choice have underlying logical contradictions. The questioning of the often arbitrary axioms can be subjective, which likely only wastes time, as the answers must be equally subjective. Objective questioning of axioms through paradoxes --- questions that can’t be answered for all answers are logically incorrect --– will uncover potentially inconsistency issues in a field's axioms that limit its scope, which when realized and corrected, will ultimately benefit the discipline. A paradox within the research on set theory, a theory regarding sets of items, had directly demonstrated how asking questions that can’t be answered can be rewarding in Mathematics. Set theory started naively with axioms that are broad and intuitive. Subsequently, mathematician Bertrand Russell posed a question that can’t be answered: ``Does the set of all sets without themselves belong in itself?'', the question being unanswerable as either yes or no created logical contradictions\citefoot{irvine_deutsch_2020}. The paradox placed theorems of naive set theory into question, providing the incentive for mathematicians to reform the axioms with the development of axiomatic set theory (ZFC) that removed this damaging paradox. Although the paradox caused knowledge setbacks in the short term as mathematicians reviewed their naive theorems to fit the new axioms, Russell’s question that could not be logically answered created knowledge in the failures of intuitive axioms for a complete set theory, as well as combining other similar logical paradoxes --- does a barber who only shaves people who don’t shave themselves shave himself? --- into questions in set theory. This generalized the theory's application in mathematics that increased the explanatory power of set theory over time. Therefore, the objective questioning of axioms led to long-term knowledge creation and reduced future similar paradoxes that can create short-term setbacks.


% analysis, % linking to question

The empirical nature of economics in the Human Sciences increases the benefits in asking questions without answers and decreases the explanatory power of answers that can’t be questioned. The knowledge in economics resides in the understanding and predicting of previous and future economic events. As the underlying human behavior fluctuates due to changing social norms and views, historical economic knowledge/answers will become incompatible to the developed world of today. Whereas the political defense of past economic answers/policies (that can't be questioned) can be harmful to a changing economy, empirically supported questions that the reigning economic belief is unable to answer creates new schools of thoughts that conforms to present-day data. The example is the handling of the 1970 United States stagflation period by the then trusted Keynesians\citefoot{nielsen_2022}, who are believers of government spending and demand-side policies in periods of recessions. The stagflation due to a oil embargo was initially handled via the methods of combating the similar 1940 Great Depression. When the economy remained in high inflation and unemployment, it naturally raised questions regarding the failure of the policy that can’t be theoretically answered. It took new monetary theorists’ new model of the economy and policies to lift the US out of stagflation. The handling and minimization of economic suffering ultimately took the questioning of a well trusted Keynesian theory, a theory that was politically unchallengeable due to its previous successes but fails under new supply-side issues. The asking of questions with empirical evidences (that the economy remained at stagflation) that the current knowledge framework is unable to answer not only created new fields of knowledge (monetarist model) that incorporates the current data, but also minimized the economic damage for US citizens that a conformation to the old Keynesian answers failed to escape. Therefore, empirical questions which can't be answered in economics are better for it ultimately discredited incorrect theories and created new knowledge that conformed to current data, which enables more effective policies.

% second point

Though, there are benefits of answers that can’t be questioned in economics for their knowledge creation speed and objectivity. Historically, economists analyzed past data to make future predictions; modern economics incorporates historical data with theoretical theories formed axiomatically. The incentives of a similar axiomatic system to that of mathematics through intuitive axioms presents a more systematic understanding of economic data and helps to make predictions that are faster and more objective than their historical counterparts. One instance is the theoretic development of the Marshall-Lerner condition in predicting a country’s net trade deficits\citefoot{mlcond}. While previous economists believed that government intervention in the exchange rates only worsens the deficit, the Marshall-Lerner condition proved theoretically that the worsening is merely short-term, and the deficit will improve over time. Without costly historical empirical data, the condition created knowledge in new economic policies through exchange rates that combat trade deficits in the long-term which were previously thought to only harm it --- this was later empirically verified through data from numerous Asian economies\citefoot{hsing_2010}. In creating theoretical answers that cannot be logically questioned, economists need not wait or to risk for empirical data (via national experiments) in order to make new objective predictive knowledge regarding the economy that can later be confirmed. Contrary, the knowledge creation by asking unanswerable questions from empirical data are slower as they are deductive. The benefit of creating theoretical inductive answers is to swiftly broaden the scope of economics as a discipline.


Ultimately, the benefits of answers that can’t be questioned versus questions that can’t be answered depends on the AOK in question. The theoretical nature of mathematics means that most truthful knowledge resides in the relationship between axioms and theorems, created by making theoretical answers that are unquestionable logically. While the knowledge within paradoxes stemming from mistakes in the axioms are present, paradoxes are fundamentally rare. In contrast, questions without answers are more beneficial in the human sciences for its empirical conformation. These questions backed by data that existing theory cannot explain offers change that brings the discipline closer to reality. The benefits of theories that can’t be logically questioned are greatly reduced --- theorems that do not correspond to reality are ultimately pointless in the human science.





\newpage
%\nocite{*}
\printbibliography

% CONLCUSION

%that are unanswerable even in a logical mathematical framework. In attacking the basis axioms of an axiomatic systems,


% NEED TWO VARIATIONS OF QUESTIONS AND ANSWERS
%Questions that questions existing theories and knowledge, and their responses.

%Questions that uncovers new knowledge through questioning new empirical discoveries, and their answers




% NOTES





% END::NOTES

% First AOK
% First point
% Against point


% Second AOK
% For point
% Against point


% Conclusion



\end{document}