\documentclass[a4paper,12pt]{article}
\usepackage[utf8]{inputenc}
\usepackage[a4paper,margin=1in,footskip=0.25in]{geometry}
\linespread{1.25}
\setlength{\parskip}{\baselineskip}


% fancy page numbers
\usepackage{fancyhdr}
% bottom right
\pagestyle{fancy}
\fancyhf{}
\fancyfoot[R]{\thepage}
\fancypagestyle{plain}{%
    \renewcommand{\headrulewidth}{0pt}%
    \fancyhf{}%
    \fancyfoot[R]{\thepage}%
}
%go
\renewcommand{\headrulewidth}{0pt}


\linespread{1.25}


\title{\vspace{-8ex}TOK Essay, Prompt 3}

\author{Terry Qi}
\date{}

\begin{document}

\maketitle
\begin{quote}
    Is it better to have questions that can't be answered than answers that can't be questioned? (Richard Feynman) Discuss with reference to mathematics and one other area of knowledge.
\end{quote}


% Thesis
Asking and answering questions are undoubtedly unique tools of mankind as they probe deeper into the mechanics and implications of the knowledge in the universe. Question arises from a sense of uncertainty towards either existing theories and knowledge frameworks or in areas with an absent of knowledge --- rhetorical questions ignored for they serve a different purpose. Answers or the act of answering are steps towards alleviating the uncertainties using logic or evidences. It is important for knowledge seekers to understand the questions to ask and the answers to make for this exchange is involved in most discipline research and it helps to rank utility of the overwhelming amount of unsolved questions posed in this era of information. Theoretical questions are hypothetical ideas built upon existing knowledge frameworks they aims to broaden the framework's scope, their answers are often unquestionable due to the theoretical nature of an answer. Empirical questions questions the validity of existing knowledge through mostly empirical discrepancies but also with logical errors; these questions are often unanswerable for their direct aggression against existing knowledge framework and only speculative answers can be posed in the short-term. With this essay, I will be exploring the utilities of theoretical unquestionable answers and empirical unanswerable questions in Mathematics and the discipline of economics from the Human Sciences. At a glance, their benefits to knowledge creation depends on the purpose of the disciplines and their view on completeness.


Theoretical questions and answers are often preferred in the knowledge framework of axiomatic mathematics for historical or practical reasons. The axiomatic system (citation) and the unquestionable answers assisted past mathematicians to both identify correct theorems out of speculative theories and to construct statements that had immense explanatory power even today (the Pythagorean theorem). Practically, the system utilizes assumptions called Axioms (citation) to which further theorems are built upon under a common system of logic. This is useful as mathematics does not require real-world connections, where knowledge purely resides in the relationships between axioms and theorems under proofs (citation). Such unquestionable relationships can greatly assist in the creation of future knowledge.

More specifically, the famous 1993 proof by the British mathematician Andrew Wiles's asserted the validity of Fermat's Last Theorem stating the absence of integer solutions of equations with form $a^n + b^n = c^n$ where $n > 2$. The proof is a combination of previous theoretical answers such as the properties and relationships between a set of special elliptical curves and modular-forms, each developed independently from one another under an axiomatic algebra system (citation). Importantly, the unquestionable theorems and the axiomatic system identified a logical mistake in the first proof that could have potentially undermined the argument. For his proof to have the same explanatory power as the knowledge it was based upon, Wiles was forced to correct the mistake in 1995 even as the result of the proof remained unchanged.

The rigorous nature of the axiomatic systems and unquestionable answers ensured only validated proofs are accepted into the knowledge area and speculative idea separated into conjectures. This creates a cascading effect of future knowledge answers inheriting the validity powers of the past. Andrew Wile's final logical proof is therefore as valid as the theorem it depended upon --- being unquestionable. The objective nature of theoretical answers serves to reduce often erroneous human intuitions in mathematical theorem creations --- a seemingly similar conjecture made by influential mathematician Leonhard Euler was disproved by counterexample through a computer search (citation). The lack of a logical proof separated Euler's conjecture from the now rigorous statement of Fermat's. The high validity of the answers that can't be questioned allowed mathematics to be science's language by providing a logical basis on which predictions and inference can be made.

That being said, empirical questions that are unanswerable have the ability to undermine entire branches of mathematics through logical paradoxes --- statements that contradict themselves. G\"odel's incompleteness theorem dictates the existence of such paradoxes and rules out an unified complete theory of mathematics (citation), providing these paradoxes with the ability to bring whole mathematical branches into question through logical inconsistencies, which in turn, weakens the subsequent theorems and conjectures based on its axioms. This exposes the weakness in the theoretical models of knowledge creation: an unquestionable answer does not always provide utility if it is built on self-contradictory axioms.

Specifically was the famous Russell's paradox in 1901 (citation)and its modern variants. It was posed in a period of mathematical trust that theoretical set theory --- the study of groups of items --- can axiomatically provide a consistence framework of classical mathematics. The paradox showed that all set theories where sets can be freely constructed must have contradictions --- creating a question that can't be answered --- and breaking an intuitive axiom that underlies naive set theories. Overtime however, the axioms were ultimately re-chosen and the paradox side-stepped by limiting the amount of sets that can be constructed.

Paradoxes are fundamentally questions against existing theories and knowledge that can't be answered; their existences in even axiomatic systems may imply that there does not exist answers that can't be question, if the action of questioning cascade towards the axioms on which the answers are based upon. For there will always be subjectivity in choosing the axioms, empirical questions are beneficial in improving the axioms chosen by placing their consistencies under question. Moreover, paradoxes likes Russell's paradox have the potential to eliminate entire branches of knowledge studies that it finds inconsistent (the study of classes, sets within themselves), exponentially worsened by the axiomatic method through its knowledge building on previous knowledges --- a fallen skyscraper due to the failures in its supports. However, this process will ultimately reduce time wasted on logically flawed pieces of knowledge.

Therefore, at least in the area of axiomatic mathematics, the unquestionablities of theoretical findings is a double edge sword: it increases the rate of knowledge development by building on the findings of others, but it falls spectacularly and definitely when a contradiction paradox is found. But rigorous axiomatic systems only survives in non-empirical areas of knowledge. On the other hand, empirical sciences based on finding the pattern within nature and humans has no luxury in being purely theoretical and axiomatic, for they must be supported by evidences and be applicable.

% ECO
Empirical questions that are unanswerable have their benefits shown in the human sciences which mangles with the unpredictability of the actions from nature and humans.



%that are unanswerable even in a logical mathematical framework. In attacking the basis axioms of an axiomatic systems,


% NEED TWO VARIATIONS OF QUESTIONS AND ANSWERS
%Questions that questions existing theories and knowledge, and their responses.

%Questions that uncovers new knowledge through questioning new empirical discoveries, and their answers




% NOTES





% END::NOTES

% First AOK
% First point
% Against point


% Second AOK
% For point
% Against point


% Conclusion



\end{document}