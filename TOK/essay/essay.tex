\documentclass[a4paper,12pt]{article}
\usepackage[utf8]{inputenc}
\usepackage[a4paper,margin=1in,footskip=0.25in]{geometry}
\setlength{\parskip}{\baselineskip}


% fancy page numbers
\usepackage{fancyhdr}
% bottom right
\pagestyle{fancy}
\fancyhf{}
\fancyfoot[R]{\thepage}
\fancypagestyle{plain}{%
    \renewcommand{\headrulewidth}{0pt}%
    \fancyhf{}%
    \fancyfoot[R]{\thepage}%
}
%go
\renewcommand{\headrulewidth}{0pt}
\usepackage[style=apa]{biblatex}
\addbibresource{essay.bib}
\usepackage{hyperref}

% Footer citing

% this only cites author if author field is present
\DeclareCiteCommand{\citeauth}
{\boolfalse{citetracker}%
	\boolfalse{pagetracker}%
	\usebibmacro{prenote}}
{\iffieldundef{author}{}{(\printtext{\printfield{author},})}}
{\multicitedelim}
{\usebibmacro{postnote}}

\newcommand{\citefoot}[1]{\footnote{\citeall{#1}}}
\newcommand{\citeall}[1]{\citeauth{#1} \citetitle{#1} (\citeyear{#1}) publisher \citelist{#1}{publisher}}

\linespread{1.67}

\title{\vspace{-8ex}TOK Essay, Prompt 3}

\author{Word Count: 1600}
\date{}

\begin{document}

\maketitle
\begin{quote}
    Is it better to have questions that can't be answered than answers that can't be questioned? (Richard Feynman) Discuss with reference to mathematics and one other area of knowledge.
\end{quote}

% TODO: Review all of the points and make them clearer


% Thesis
Asking and answering questions are undoubtedly unique tools of mankind as they probe deeper into the mechanics and implications of the knowledge in the universe. Questions arise from a sense of uncertainty towards either existing theories and knowledge frameworks or in areas with the absent of knowledge --- rhetorical questions ignored for they serve a different purpose. Answers or the act of answering are steps towards alleviating the uncertainties using logic or evidences. It is important for knowledge seekers to understand the questions to ask and the answers to make for this exchange is involved in most discipline research and helps to rank utilities of the overwhelming amount of unsolved questions posed in this era of information. Theoretical questions are hypothetical ideas built upon existing knowledge frameworks aiming to broaden the framework's scope, their answers are often unquestionable due to the theoretical nature of an answer. Empirical questions questions the validity of existing knowledge through mostly empirical discrepancies but also with logical errors; these questions are typically unanswerable or can be only speculative for their empirical and logical evidences against existing knowledge framework. With this essay, I will be exploring the utilities of theoretical unquestionable answers and empirical unanswerable questions in Mathematics and the discipline of economics from the Human Sciences. At a glance, their benefits to knowledge creation depends on the purpose of the disciplines and their view on completeness.


Theoretical questions and answers are preferred in the knowledge framework of axiomatic mathematics for historical or practical reasons. The axiomatic system\citefoot{axiomatic} and the unquestionable answers assisted past mathematicians to both identify correct theorems out of speculative theories and to construct statements that had immense explanatory power even today (the Pythagorean theorem). Practically, the system utilizes assumptions called Axioms\citefoot{axiom} to which further theorems are built upon under a common system of logic. This is useful as mathematics does not require real-world connections, where knowledge purely resides in the relationships between axioms and theorems under proofs\citefoot{horsten_2022}. Such unquestionable relationships can greatly assist in the creation of future knowledge.

More specifically, the famous 1993 proof by the British mathematician Andrew Wiles's asserted the validity of Fermat's Last Theorem stating the absence of integer solutions of equations with form $a^n + b^n = c^n$ where $n > 2$. The proof is a combination of previous theoretical answers such as the properties and relationships between a set of special elliptical curves and modular-forms, each developed independently of one another under an axiomatic algebra system\citefoot{sakarda_tan_tipirneni_2020}\citefoot{masdeu_2015}. Importantly, the unquestionable theorems and the axiomatic system identified a logical mistake in the first proof that could have undermined the argument. For his proof to have the same explanatory power as the knowledge it was based upon, Wiles was forced to correct the mistake in 1995 even as the result of the proof remained unchanged.

The rigorous nature of the axiomatic systems and unquestionable answers ensured only validated proofs are accepted into the knowledge area and speculative idea separated into conjectures. This creates a cascading effect of future knowledge answers inheriting the validity powers of the past. Andrew Wile's final logical proof is therefore as valid as the theorem it depended upon --- being unquestionable. The objective nature of theoretical answers serves to reduce often erroneous human intuitions in mathematical theorem creations --- a seemingly similar conjecture made by influential mathematician Leonard Euler was disproved by counterexample through a computer search\citefoot{lander_parkin_1966}. The lack of a logical proof separated Euler's conjecture from the now rigorous statement of Fermat's. The high validity of the answers that can't be questioned allowed mathematics to be science's language by providing a logical basis on which predictions and inference can be made.

That being said, empirical questions that are unanswerable have the ability to undermine entire branches of mathematics through logical paradoxes --- statements that contradict themselves. G\"odel's incompleteness theorem dictates the existence of such paradoxes and rules out an unified complete theory of mathematics\citefoot{bellos_2022}, providing these paradoxes with the ability to bring whole mathematical branches into question through logical inconsistencies, which in turn, weakens the subsequent theorems and conjectures based on its axioms. This exposes the weakness in the theoretical models of knowledge creation: an unquestionable answer does not always provide utility if it is built on self-contradictory axioms.


Specifically was the famous Russell's paradox in 1901\citefoot{wiles_1995} and its modern variants. It was posed in a period of mathematical trust that theoretical set theory --- the study of groups of items --- can axiomatically provide a consistence framework of classical mathematics. The paradox showed that all set theories where sets can be freely constructed must have contradictions that contradicted an intuitive axiom that underlies naive set theories. The axioms were ultimately re-chosen and the paradox side-stepped by limiting the amount of sets that can be constructed.

Paradoxes are fundamentally questions against existing theories and knowledge that can't be answered; their existences in even axiomatic systems may imply that there does not exist answers that can't be questioned. For there will always be subjectivity in choosing the axioms, empirical questions are beneficial in improving the axioms chosen by placing their consistencies under question. Moreover, paradoxes likes Russell's paradox have the potential to eliminate entire branches of knowledge studies that it finds inconsistent, exponentially worsened by the axiomatic method through its knowledge building on previous knowledges analogous to a falling skyscraper from the failures in its supports. The questioning of axioms will ultimately reduce time wasted on logically flawed pieces of knowledge.

Therefore, at least in the area of axiomatic mathematics, the unquestionablities of theoretical findings is a double edge sword: it increases the rate of knowledge development by building on the findings of others, but it falls spectacularly and definitely when a contradiction paradox is found. Yet, rigorous axiomatic systems only survives in non-empirical areas of knowledge. On the other hand, empirical sciences based on finding knowledge within nature and humans has no luxury in being purely theoretical and axiomatic, for they must be supported by evidences and be applicable.

% ECO
Empirical questions that are unanswerable becomes more beneficial in human sciences that mangle with the unpredictable humans social structures and actions. Disciplines like economics aim to model human behavior in consumption and production, of which future economic predictions and suggestions are made upon\citefoot{wikipedia_2021}. Importantly, all theoretical conjectures must match empirical evidences provided through past data collection. For human behavior is complex and computationally expensive to model\citefoot{pentland_liu_1999}, a complete theoretical system is impossible with the technologies of today --- often seen are statistical judgments based on probabilities. Unanswerable empirical questions supported by new observations that questions existing theories can therefore filter theoretical speculative ideas and create knowledge in improving future predictions.

It is common to see periods of theoretical economical revolutions sparked by new observations and questions which are unanswerable. One such case is the US economic stagflation in the 1970s. Due to an oil embargo, the US suffered an unprecedented period of high inflation and low employment that surprised the reigning Keynesian economists at the time\citefoot{nielsen_2022}. It heavily contrasted against the standard theory of an inverse relationship between inflation and unemployment that raised questions in the abilities of governments in controlling the economy. This new empirical evidence eventually prompted new monetarist theories lead by Milton Friedman that raised the US economy out of recession.

The previously unanswerable empirical questions based on new observation also had the ability to create new knowledge that can help future predictions. A modern case is the governments' reuse of past stagflation knowledge towards the high inflation low economic growth periods of the COVID-19 pandemic for its similarities. If governments had adhered to existing theories of theoretical answers by restricting opposing views, it risks extending the inflationary period and reducing trust in the economic discipline that damages future funding towards social knowledge creation. A destructive case resides in the Turkey president's unorthodox theoretical economic views that led to public concerns and rising poverty rates\citefoot{guardian_2022}. Therefore, the lack of empirical questioning have the ability to reduce knowledge trustworthiness and with wide-reaching effects on unaffiliated knowers.

Yet the basis of a theoretical axiomatic structure remains tempting even in an empirical human science for its knowledge derivative power. A theory's basis on theoretical answers that can't be questioned will increase the rate of knowledge development in rapid predictions of potential knowledge without costly and time consumptive experiments or historical evidences. One instance is the theoretical development and proof of the Marshall-Lerner condition in predicting net trade deficits as a function of time\citefoot{mlcond}. The condition's concrete foundations on trade elasticities and differential calculus resembles that of the axiomatic method brings certainty. The subsequent empirical confirmations\citefoot{hsing_2010} reveals the potential benefits in the mixture of logical hypothesis (unquestionable answers) and observation (unanswerable statements) used in the scientific method, in that new knowledge is swiftly created through an unquestionable theoretical finding, then confirmed via observations previously unanswerable.

In summary, the utilities of unquestionable theoretical answers against unanswerable empirical questions depends on the area of knowledge in which the questions and answers are brought within. Theoretical answers are often preferred in mathematics for its lack of empirical observations allows the knowledge within to withstand the test of time, but is vulnerable to occasional unavoidable paradoxes barring off branches of knowledge; Empirical questions that are unanswerable are helpful in recognizing the limitations of theoretical knowledge in the human sciences and brings the knower closer to the truth through questioning incorrect logical answers. However, the scientific method's mixture of a theoretical basis with empirical questioning can increase the rate of knowledge creation and is ultimately a commonly accepted knowledge structure.


\newpage
%\nocite{*}
\printbibliography

% CONLCUSION

%that are unanswerable even in a logical mathematical framework. In attacking the basis axioms of an axiomatic systems,


% NEED TWO VARIATIONS OF QUESTIONS AND ANSWERS
%Questions that questions existing theories and knowledge, and their responses.

%Questions that uncovers new knowledge through questioning new empirical discoveries, and their answers




% NOTES





% END::NOTES

% First AOK
% First point
% Against point


% Second AOK
% For point
% Against point


% Conclusion



\end{document}