\documentclass[a4paper,12pt]{article}
\usepackage[utf8]{inputenc}
\usepackage[a4paper,margin=1in,footskip=0.25in]{geometry}
\setlength{\parskip}{\baselineskip}

\usepackage{amsmath}


% fancy page numbers
\usepackage{fancyhdr}
% bottom right
\pagestyle{fancy}
\fancyhf{}
\fancyfoot[R]{\thepage}
\fancypagestyle{plain}{%
    \renewcommand{\headrulewidth}{0pt}%
    \fancyhf{}%
    \fancyfoot[R]{\thepage}%
}
%go
\renewcommand{\headrulewidth}{0pt}
\usepackage[style=apa]{biblatex}
\addbibresource{essay.bib}
\usepackage{hyperref}

% Footer citing

% this only cites author if author field is present
\DeclareCiteCommand{\citeauth}
{\boolfalse{citetracker}%
	\boolfalse{pagetracker}%
	\usebibmacro{prenote}}
{\iffieldundef{author}{}{(\printtext{\printfield{author},})}}
{\multicitedelim}
{\usebibmacro{postnote}}

\newcommand{\citefoot}[1]{\footnote{\citeall{#1}}}
\newcommand{\citeall}[1]{\citeauth{#1} \citetitle{#1} (\citeyear{#1}) publisher \citelist{#1}{publisher}}

\linespread{1.67}

\title{\vspace{-8ex}TOK Essay, Prompt 3}

\author{Word Count: 1700}
\date{}

\begin{document}

\maketitle
\begin{quote}
    Is it better to have questions that can't be answered than answers that can't be questioned? (Richard Feynman) Discuss with reference to mathematics and one other area of knowledge.
\end{quote}

The asking and answering of questions are undoubtedly unique tools of mankind as they probe deeper into the mechanics and implications of knowledge in the universe. Questions arise from the uncertainty towards either existing theories and frameworks against disproving evidences or within the implication of the knowledge. Answers are arguments that alleviates the uncertainties posed by the question through logic or further evidences. Some answers can’t be questioned, for either they are derived by proof from axioms, or are protected and enforced by society. Some questions can’t be answered, from a lack of empirical evidences or the logical impossibility for the answers through paradoxes. It is important for knowledge seekers to understand the type of questions to ask and the type of answers to make in different knowledge areas for not only is this exchange the main method of knowledge creation, and optimizing it will increase rate of knowledge growth in scope and depth, but also to prevent affecting others through decisions not questioned. In this essay, the exploration of the utilities between answers that can't be questioned against questions that can't be answered within Mathematics and the discipline of economics from the Human Sciences points to the perspective that it depends on the discipline the question or answer is raised in.

%With this essay, I will be exploring the utilities and preferences of answers that can’t be questioned against questions that can’t be answered in Mathematics and the discipline of economics from the Human Sciences.

Due to the theoretical nature of Mathematics – where knowledge resides in the provable relationships between theorems and axioms – answers that can’t be questioned are better for the discipline as it enforces logical results that provides certainty in the area of knowledge. The filtering of answers that must not be questionable separates less important hypotheses from theorems with high correctly and applicable theorems. On the other hand, the searching and attempting of questions that can’t be answered – due to absurdity or questions towards Axioms, statements assumed to be correct – provides little impact and can even be dangerous as it can reduce time and resources spent on knowledge building by answering relevant questions. Historically, the axiomatic method generated powerful theorems that we use today (such as the Pythagorean theorem regarding triangles), but specifically, the case of two similar hypothesis demonstrated the benefits of a convention to build knowledge through answers that can’t be questioned. Fermat's hypothesis suggested a lack of solutions for a specific polynomial equation, namely:
\[
    a^n + b^n = c^n, \, \text{where} \,\, a, b, c \in \mathcal{Z} \quad n \ge 3
\]
Similarly, Euler’s conjecture – a generalization of Fermat’s hypothesis – also suggested a lack of solutions for a generalized polynomial equation, namely:
\[
    a_1^k+a_2^k + ... + a_n^k = b^k, \, \text{where} \,\, a_i, k \in \mathcal{Z} \quad n \ge k
\]
As unquestionable answers in the form of proofs are required to convert conjectures to theorems, the statements, albeit seemly true for a large number of cases, are not accepted as absolute truth for around 250 years. Interestingly, while Fermat's hypothesis was eventually proved by Andrew Wiles based upon existing foundational theorems, the answer that is the proof creating new fields of study for its confirmation of other dependent conjectures; the conjecture by Euler was disproved by a counterexample by a computer simply trying a large amount of numbers. In essence, this dual example displayed how the truthfulness of even similar formed statements can differ. It demonstrated how without the proofs between axioms (and their corollary) that can’t be questioned as the basis of knowledge, human intuition may view Euler’s conjecture just as correct as Fermat's, therefore creating erroneous information in the Mathematics. The acceptance of Euler’s conjecture and general mathematics without logical proofs encourages an unfruitful exploration based on fundamentally false information and the unfruitful questioning of existing results. Contrast to the research and exploration of only proved results that cannot be questioned for its logical basis, the knowledge seeker can trust that their contributions through the same axiomatic system have the same validity and is as unquestionable as existing results. The focus on proofs – answers that can’t be questioned – ultimately enforces rigor in mathematical knowledge that reduces wasted time and resources spent on unrewarding questionable fields.

Yet, the exploring and asking of questions that can’t be answered can still provide knowledge particularly on the limits of an Axiomatic System – a system of theorems and proofs based on unique set of axioms. Although corollary by proofs are answers that cannot be questioned logically, the selection of the subjective axioms definitely can if the choice have underlying logical contradictions. The questioning of the often arbitrary axioms can be subjective, which likely only wastes time for they can’t be answered because all answers must also be subjective and are equally correct; objective questioning of axioms through paradoxes – questions that can’t be answered for all answers are equally incorrect logically – can uncover the problems within axioms that limits the scope of an Axiomatic System, that in the long-term, will ultimately benefit the discipline in knowledge gain. A paradox within the research on set theory, a theory regarding sets of items, had directly demonstrated how asking questions that can’t be answered can be rewarding in Mathematics in the long-term. Set theory started naively without axioms restricting possible items in  sets, for the mathematicians failed to intuitively think of its issues. Yet subsequently, Russell posed a question that can’t be answered: “Does the set of all sets without themselves belong in itself?”, the question being unanswerable as either yes or no created logical contradictions. The paradox placed the axioms (and the lack of) of naive set theory into question, providing incentives for mathematicians to reform the axioms with the development of axiomatic set theory (ZFC) that removed the existence of such sets – therefore eliminating the paradox. Although the paradox caused knowledge setbacks in the short term as mathematicians reviewed their naive theorems to fit the new axioms, Russell’s question that could not be logically answered created knowledge in the limits required for a complete set theory, as well as combining other similar logical paradoxes – does a barber who only shaves people who doesn’t shave themselves shave himself? - into questions in set theory, generalizing the topic and creating a revolution of infinity in mathematics. Therefore the objective questioning of axioms led to long-term knowledge creation and reducing future similar paradoxes that can cause similar short-term setbacks.
% analysis, % linking to question


The empirical conformation of economics in the human sciences increases the benefits in asking questions without answers and decreases the explanatory power of answers that can’t be questioned. As the knowledge in economics resides in the understanding and predicting of previous and future economic behavior, and the underlying human behavior under constant fluctuation due to changing social norms and views, historical economic knowledge/answers will become incompatible to the developed world of today. Empirically supported questions that the reigning economic belief is unable to answer is required for new schools of thoughts that conforms to present-day data to replace the old of the past, with the political defense of past economic answers/policies can be harmful to the changed economy. The example is the erroneous handling of the 1970 US stagflation period by the then trusted Keynesians, who are believers of government spending and demand-side policies in periods of recessions. The stagflation due to a supply shortage of imported oil was initially handled via the methods combating the 1940 Great Depression. When economy remained in high inflation and unemployment, it naturally raised questions regarding to the failure of the policy that can’t be objectively answered. It took new monetary theorists’ new model of the economy and policies to lift the US out of stagflation. The handling and minimization of economic suffering ultimately took the questioning of a well trusted Keynesian theory, a theory that was politically unchallengeable due to its previous successes but fails under new supply-side issues. The asking of questions with empirical evidences (that the economy remained at stagflation) that the current knowledge framework is unable to answer not only created new fields of knowledge (monetarist model) that incorporates the current data, but also minimized the economic damage in the form of persistent high inflation and unemployment to US citizens that old Keynesian policies failed to escape. Questions that can’t be answered increased the explanatory power of economics when past answers fails, marking its importance.
% second point

Though there are benefits of theoretical answers that can’t be logically questioned for their swift and predictive knowledge creation. Although historically economists analyzed past data to make future predictions, modern economics also contains theoretical concepts. The incentives of the creation of a similar axiomatic system through minimal intuitive axioms (citations) to that of mathematics is a more systematic understanding of economic data and to make predictions that are faster and more objective than historical predictions. One instance is the theoretic development of the Marshall-Lerner condition in predicting a country’s net trade deficits. While previous economists believed that government intervention in the exchange only worsens the deficit, the condition proved theoretically that the worsening is merely short-term, and the deficit will improve overtime. Even without historical empirical data, the condition created knowledge in new economic policies through exchange rates to combat trade deficits in the long-term that were previously thought to only harm it – this was later empirically verified through data from numerous Asian economies. In creating theorems/answers that cannot be logically questioned, economists need not to wait or to risk for empirical data (via experiments) to create new predictive knowledge regarding the economy. Contrary, the knowledge creation by asking questions unanswerable due to empirical data are slower for they are deductive. The inductive benefit of creating answers is to swiftly broaden the scope of economics as a discipline.

Ultimately, the benefits of answers that can’t be questions versus questions that can’t be answers depends on the AOK in question. The theoretical nature of mathematics means that most knowledge resides in the relationship between axioms and theorems, created by making theoretical answers that are unquestionable logically. While the knowledge within paradoxes stemming from mistakes in the axioms are present, paradoxes are fundamentally rare. In contrast, questions without answers are more beneficial in the human sciences for its empirical conformation. These questions backed by data that existing theory cannot explain offers change that brings the discipline closer to reality. The benefits of theories that can’t be logically questioned are greatly reduced as theorems that does not correspond to reality are ultimately useless in the human science.



\newpage
%\nocite{*}
\printbibliography

% CONLCUSION

%that are unanswerable even in a logical mathematical framework. In attacking the basis axioms of an axiomatic systems,


% NEED TWO VARIATIONS OF QUESTIONS AND ANSWERS
%Questions that questions existing theories and knowledge, and their responses.

%Questions that uncovers new knowledge through questioning new empirical discoveries, and their answers




% NOTES





% END::NOTES

% First AOK
% First point
% Against point


% Second AOK
% For point
% Against point


% Conclusion



\end{document}