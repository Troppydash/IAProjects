\documentclass[a4paper,12pt]{article}
\usepackage[utf8]{inputenc}
\usepackage[a4paper,margin=1in,footskip=0.25in]{geometry}
\linespread{1.25}
\setlength{\parskip}{\baselineskip}


% fancy page numbers
\usepackage{fancyhdr}
% bottom right
\pagestyle{fancy}
\fancyhf{}
\fancyfoot[R]{\thepage}
\fancypagestyle{plain}{%
    \renewcommand{\headrulewidth}{0pt}%
    \fancyhf{}%
    \fancyfoot[R]{\thepage}%
}
%go
\renewcommand{\headrulewidth}{0pt}


\linespread{1.25}


\title{\vspace{-8ex}TOK Essay, Prompt 3}

\author{Terry Qi}
\date{}

\begin{document}

\maketitle
\begin{quote}
    Is it better to have questions that can't be answered than answers that can't be questioned? (Richard Feynman) Discuss with reference to mathematics and one other area of knowledge.
\end{quote}


% Thesis
Asking and answering questions are undoubtedly unique tools of mankind as they probe deeper into the mechanics and implications of the knowledge in the universe. Question arises from a sense of uncertainty towards either existing theories and knowledge frameworks or in areas with an absent of knowledge --- rhetorical questions ignored for they serve a different purpose. Answers or the act of answering are steps towards alleviating the uncertainties using logic or evidences. It is important for knowledge seekers to understand the questions to ask and the answers to make for this exchange is involved in most discipline research and it helps to rank utility of the overwhelming amount of unsolved questions posed in this era of information. Theoretical questions are hypothetical ideas built upon existing knowledge frameworks they aims to broaden the framework's scope, their answers are often unquestionable due to the theoretical nature of an answer. Empirical questions questions the validity of existing knowledge through mostly empirical discrepancies but also with logical errors; these questions are often unanswerable for their direct aggression against existing knowledge framework and only speculative answers can be posed in the short-term. With this essay, I will be exploring the utilities of theoretical unquestionable answers and empirical unanswerable questions in Mathematics and the discipline of economics from the Human Sciences. At a glance, their benefits to knowledge creation depends on the purpose of the disciplines and their view on completeness.


Theoretical questions and answers are often preferred in the knowledge framework of axiomatic mathematics for historical or practical reasons.





% NEED TWO VARIATIONS OF QUESTIONS AND ANSWERS
%Questions that questions existing theories and knowledge, and their responses.

%Questions that uncovers new knowledge through questioning new empirical discoveries, and their answers




% NOTES





% END::NOTES

% First AOK
% First point
% Against point


% Second AOK
% For point
% Against point


% Conclusion



\end{document}