\documentclass[a4paper,12pt]{article}
\usepackage[utf8]{inputenc}
\usepackage{graphicx}
\usepackage{float}

%opening
\title{TOK: Are some things unknowable?}
\author{Terry Qi}

\begin{document}

\maketitle
\begin{enumerate}
 \item Law of motion
 \item Double Pendulum
 \item Progress bars
\end{enumerate}

\begin{figure}[h!]
 \centering
 \[
    F = ma
 \]
 \caption{Newton's law of motion}
\end{figure}

\begin{figure}[h!]
 \centering
 \includegraphics[scale=0.25]{dpend.png}
 \caption{Three Double Pendulums}
\end{figure}

\begin{figure}[h!]
 \centering
 \includegraphics[scale=0.35]{progress.png}
 \caption{A progress bar}
\end{figure}



\newpage

% knowable thing but could be failified
\paragraph{1}

% knowable thing but is chaotic
\paragraph{2}

% unknowable thing that expresses certainty
\paragraph{3}

\end{document}
