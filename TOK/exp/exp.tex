\documentclass[a4paper,11pt]{article}
\usepackage[a4paper,margin=1in,footskip=0.25in]{geometry}
\usepackage[utf8]{inputenc}
\usepackage{graphicx}
\usepackage{float}
\usepackage{parskip}
\usepackage[style=apa]{biblatex}
\addbibresource{exp.bib}
\usepackage{hyperref}


\title{What challenges are raised by the dissemination and/or communication of knowledge?}
\author{Terry Qi}

\begin{document}

\maketitle

Word Count: 942

Objects:
\begin{enumerate}
 \item The law of demand within my economics book
 \item The bible according to Spike Milligan
 \item Google search result of Fredrick Douglass
\end{enumerate}

\newpage

% Uncertainty
% Context
% Oversaturation

% intro
% This exhibition will explore the prompt by reflecting upon the knowledge within the sciences, more specifically answering whether uncertainty hinder the acceptance and communication of knowledge. While mathematicians often take deductive proofs as granted, this is less common in the area of sciences. Every experiment contains empirical results, but depending on different people with their own personal backgrounds, beliefs, ideologies, or perspectives, the significance of the result will differ between the members of the public. More directly, the lack of trust in the sciences may be a disbelief towards its empirical methods, that uncertainty is untrustworthy and does not bring knowledge.


% knowable thing but could be failified
\subsection*{Object \ref{fig:lod}, The law of demand within my economics book}

\begin{figure}[H]
 \centering
 \includegraphics[scale=0.3]{ecobook.png}
 \caption{The law of demand as stated in the book \parencite{blink_dorton_2012}}
 \label{fig:lod}
\end{figure}

This formula comes from my economics coursebook, conveying the economical idea that people will buy more products when prices are lower. This is widely controversial within our class for its inclusion of the word ``law'' and is also debated widely particularly among scientists \parencite{DKahneman}. The presentation of this ``law'' complicates people's perception of its underlying knowledge when disseminated.

A designated name can difficult the attempt of conveying a piece of knowledge.
The use of the word ``law'' displays certainty --- arguing human behavior to be very predictable that it never fails, while it is meant to convey a specific case of empirical correlation under strict assumptions. This often confuses learners in understanding the theory, especially during skimming when the law is disseminated  --- its name is an oxymoron in communicating the assumptions for applying the theory. Coming from a maths background, I initially took the word literally --- that it always applies, which is against its underlying knowledge. Furthermore, the detailed textbook explanations communicate using uncertain terms like ``often'' and ``usually''.
%the explanation of the law as shown in the object further complicates this matter by contradicting to the definition of a law: ``a statement of fact'' by using the terms ``often'' and ``usually''.
These words send more contradictory meanings of this economic idea to the learner: this is a law, but it sometimes happens. They all contribute to the ambiguity challenge in knowledge communications because of wording.

This object is included in the expedition for it had affected my personal learning --- only after a detailed discussion with the teacher and further research on the topic did I understand its true meaning. The object also connects the broad naming issues in the sciences --- naming the existence of charge ``negative''. Not only does this harm the communication of essential knowledge, correcting these errors are often very difficult for consistency reasons. This showcases the challenges that languages pose in the distribution of knowledge.

% The natural scientists' critique only serves to reveal the flaws in their ``scientific methods'', that the entirety of physics is in essence, fitting numbers to interpretations. While both parties disagrees in acknowledging the shakiness of their disciplines, they are similar in framing the confusion as an ``interesting property'' and a benefit of the sciences, ultimately displaying how the interpretation of empirical evidence differs depending on your personal beliefs and backgrounds.

% context
\subsection*{Object \ref{fig:bible}, The bible according to Spike Milligan}

\begin{figure}[H]
 \centering
 \includegraphics[scale=0.07]{bible.jpg}
 \caption{The bible according to Spike Milligan \parencite{milligan_2016}}
 \label{fig:bible}
\end{figure}

This object is Spike Milligan's parody of the old testament. It modernizes the events within the bible and exaggerates the supernatural events with humor. Although I view the bible to be out of date and thus hard to trust its messages and recommendations, this book managed to cross such barriers and communicated the knowledge within the bible. This implies how changes in era and culture present a challenge of mismatching contexts in the distribution of knowledge.


%The object is a book issued to me by my friend out of interest, for it parodies the old testament of the bible. The holy bible itself is interesting for its distinctive context
%--- there are often disagreements on the origin of the bible. Some, and I, use this fact to justify our disbeliefs of the information within the bible.
%--- all of the stories are set in an ancient era. This had deterred me from reading and understanding the book. However, for Spike Milligan had only modified its context while retaining the information the bible held, it is interesting that I find myself enjoying its stories and tales.
The title and content of this book present solutions to the challenges of a loss of context in transferring Christian knowledge and outline them. The title of the book, ``...according to Spike Milligan'', showcases the requirements of names on knowledge in the current era. A book without an author is often difficult to trust and connect to for we find it uncertain in credibility --- a consequence of the wide availability of information modern humans are exposed to. This shows the different human expectations of forms of knowledge over time, presenting the challenge of context that Milligan attempted and succeed in solving. Its evidence lies in the popular reviews of the object even among non-religious readers, myself included \parencite{Review}.

Milligan's presentation of the creation story also shows the challenge of relatability, or connections in the transfer of knowledge. It relates the various days of creation to UK specific jokes: mocking the difficulty of the GCES with God correctly identifying ``night''. There is an attempt to communicate to the reader in an understandable, relatable context, while maintaining truth to biblical events. I find this to be resolving the challenges due to a change in the form of knowledge from preachers to books, which is very significant especially in the translation of knowledge. This object has demonstrated the challenges in communicating ideologies over cultures, and their dissemination over time, hence its place as my second object.


%The positive personal experience showcases how context can change the ability in the dissemination of knowledge. The existence of a definite context that is set in the modern era helps the reader to visualize the events, which connected to me using modern language and references: such as the mentioning of the GSES test. In contrast with the bible's outdated ancient settings, the object is more easily understood by the modern audience. It also serves to communicate the context of the bible, serving as a base of christian knowledge that I can build upon through my newfound interest on the bible stories. This showcase the importance of context in one's understanding of knowledge, for the parody received even positive reviews online from a mainly non-christian audience \parencite{Review}.

% When people think about the bible, they think of the connection and quotations within it rather than the plausibility and validity of the work. Historically speaking, the bible is special in its unknown origin and writer. Some people view this uncertainty as an argument to reject all meanings within the book, some accepts the knowledge it presents and not by its origin, and some values both interpretations equality.


%The parody of the bible is intriguing for it beautifully shows how the context of its origin can be completely different while retaining its messages and knowledge. The first chapter references the creation of the world, but from a perspective of a British startup firm. Unexpectedly, this shows that the messages of the bible --- that God is took seven days to create the world, is independent of its context, showcasing how the uncertainty of the origin can have little effects on the knowledge presented.

%Therefore I included this parody of the bible in the exhibition to be an example where context helps the communication of knowledge. Furthermore, it displays the inevitability of biases in the form of context within the distribution the knowledge, for a single piece of work can be rephrased to appeal to a wider audience. This implies the possibility for knowledge to be distorted or even lost overtime due to context --- a realistic example of a phenomenon of ``Chinese Whispers''.

% Judging from the positive reviews from the various purchasers and I, it seems to display the ability of a common ideology or pure interest to decrease the significance of the context of knowledge. This is in analogy to the ideas presented by the economists --- that the idealistic axioms are justified by a significant result in the knowledge that it creates.


% The object is my personal simulation of some swinging double pendulums. On the surface, it seems that the advances in the natural sciences should be able predict its motions to the atom. But the discovery of chaos questions the fundamental values of physics as a whole. In a sense, some people consider the possibility of uncertainty to be roadblock in the acquiring of knowledge in the sciences.

%Physics as a discipline follows a trend of experiment, theories, and prediction to formulate our understanding of the universe.

% unknowable thing that expresses certainty
\subsection*{Object \ref{fig:download}, Google search result of Fredrick Douglass}

\begin{figure}[H]
 \centering
 \includegraphics[scale=0.25]{douglass.png}
 \caption{Google search result of Frederick Douglass \parencite{frederick}}
 \label{fig:download}
\end{figure}


The object here is my personalized Google search result for Frederick Douglass. The invention of search-engines and the internet had brought entire catalogues of encyclopedias to devices that you can hold in your hand. But the easiness of information can also be a challenge in the distribution of knowledge, as one can be overwhelmed with millions of sources (shown in the object with 22 million results). This often creates a heuristic bias favoring easy information rather than correct information on the internet.

The object has the website Wikipedia as the top result, which displays the internet's effects on the sharing of information online. The ordering of knowledge sources limits the reach of other lesser popular knowledge sources further down the page, as the average viewer is likely to select the first result. This creates a challenge in the dissemination of knowledge with a bias towards popular sources. The single source of information narrows the scope of human knowledge --- students often conduct research based on Wikipedia alone, and I have heavily relied on this page for my English research and didn't bother finding other perspectives on his life. It shows how people lose decision-making abilities in their selection of information on the internet as a consequence of the masses of knowledge available online, causing most events to be only seen from one side of the argument.

 %People in the 21st century would often pick the most visible and easiest route to get information and knowledge on a subject, creating a bias towards a single source of truth stemming from the effects of an oversaturation of knowledge.

The importance of the internet in the modern age justifies this object's position in my exhibition. Its statistics on the number of results made me realize the true difficulty in understanding a topic from all views on the internet, which poses a challenge for me to find all the perspectives on the speaker. Moreover, the sea of information on the internet may imply a future where knowledge does not exist, a world full of only opinions and persuasive writings due to the laziness or inability of humans to check all trillion sources on a single event, displaying the hypothetical peek of the challenges in the communication of truth. Sometimes the fewer decisions the better, and the less information, the more meaningful they are.

%The object is a progress bar of a file download in the my Firefox browser. It have the purpose of displaying the time it takes to download this very important file I need. This sight had only became more common in the information era, yet it had surprised me to know that the underlying progress it displays has no certainty. The progress bar serves as a distraction to the nondeterministic information of file downloading for the user.

%While this little bar can come in many forms, there exist an endless flow of people complaining the inaccuracy of the device, yet the purpose of the device is for good --- to create feedback upon an action. It implies the idea that covering uncertainty with certainty, where knowledge is deliberately hidden, is undesirable to people. Additionally, versions of progress bars which does not cover the uncertainty does not escape the criticisms. It seems to me that people fear the concept of covering uncertainty with certainty, showcasing inability to reduce uncertainty in the communication of knowledge.

%The common appearances (and criticisms it faces) of the item helps in broadening the scope of the knowledge question. The progress bar showcases the general knower's desire for information in the event of a knowledge barrier, and its fear for the censoring of knowledge. For I have made progress bars myself in my apps, I find the understanding of the underlying knowledge the bar covers --- whether uncertain or not, helps in reducing the frustration of such a progress bar. This showcases the ability of uncertainty, even if covered with certainty, to induce frustration and communication of knowledge.

\newpage
\printbibliography

% REFERENCE BOOK, GOOGLE, WIKIPEDIA PAGE, AND BOOK


\end{document}
